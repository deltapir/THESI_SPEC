\begin{abstract}
L'Azienda Ospedaliera Universitaria ``Federico II'' è uno dei poli di riferimento in ambito ospedaliero e universitario di tutto il Mezzogiorno e rappresenta uno dei centri di più elevata specializzazione e qualificazione sul territorio nazionale. Operativa già dal 
1972, è costruita secondo uno schema a padiglioni che al giorno d'oggi risulta ancora completamente conservato. Le logiche e le idee alla base della sua progettazione, nonché della conseguente costruzione, ci sono state tramandate dall'\tit{Ing.}{Corrado Beguinot}, che all'epoca fu il direttore dei lavori. Con il procedere degli anni e con l'avvento di normative sempre più stringenti in ambito energetico, risulta evidente, se non scontato, riqualificare gli impianti meccanici e la struttura edilizia. In questo senso, il presente elaborato di laurea vuole essere un'analisi progettuale di un intervento di riqualificazione energetica da effettuarsi sull'edificio 2 di cardiochirurgia dell'AOU ``Federico II''.

%Tra tutte le strutture civili, quelle ospedaliere sono fra le più energivore. La causa principale va ricercata nelle particolari attività svolte, che richiedono grandi quantità di energia per poter garantire la miglior qualità di servizio agli utenti e per far fronte all'importante domanda di energia elettrica richiesta dalle apparecchiature e dagli strumenti diagnostici. Un'altra importante voce che incide sul bilancio energetico di un ospedale è costituita dalla climatizzazione degli ambienti, legata al raggiungimento e mantenimento delle elevate qualità dell'aria richieste per svolgere le attività sanitarie, si pensi alle particolari condizioni di asetticità richieste nei reparti operatori o nei locali dove sono assistiti i pazienti con patologie critiche. Occorre inoltre considerare l’approccio con cui
%molti degli ospedali oggi presenti nel nostro Paese sono stati costruiti. Nei decenni precedenti, grazie ai bassi costi dell'energia e soprattutto ad una minore sensibilità politica e sociale verso la sostenibilità economica ed ambientale delle attività umane, la progettazione e costruzione delle strutture sanitarie era orientata al raggiungimento degli standard sanitari richiesti, trascurando l’efficienza del sistema edificio-impianto.

%Da queste premesse appare evidente come i bilanci energetici degli ospedali, e più in generale delle strutture sanitarie, presentino elevati consumi elettrici e termici i cui costi ricadono nei bilanci economici delle aziende ospedaliere, e di conseguenza, vista l’elevata incidenza dei costi della sanità pubblica, sulle tasche dei cittadini/contribuenti. In questi tempi di crisi economica, di fronte alla sempre più impellente necessità di contenere la spesa pubblica e costi dei servizi ai cittadini, senza intaccare il livello qualitativo delle prestazioni erogate, la gestione energetica degli ospedali è una tra le innumerevoli voci di spesa la cui razionalizzazione è di fatto obbligatoria.
\end{abstract}