\chapter{La descrizione del bando e teoria sulla climatizzazione degli edifici}
\thispagestyle{empty}
\texttt{Vedi al limite di mettere questo capitolo dopo quello riguardante la storia del policlinico.
	Prendi degli stralci del bando, sottolineando le parti in cui si specifica la combinazioni di impianti da realizzare. 
	Alla fine di questo capitolo metti gli obiettivi della tesi. <-- al limite vedi di inserire questa parte anche nel sommario. Siccome questa parte è introduttiva della tesi (in quanto vengono spiegati gli obiettivi etc etc) vedi di usare questo capitolo come introduttivo o una specie di Introduzione alla tesi prima del primo capitolo vero e proprio.
	Parla pure dei softwares usati.\\ Vedi di inserire in questo capitolo la descrizione dei software e della fisica}\\
\vspace{2cm}

L'essere umano ha il continuo desiderio di vivere un luogo in condizioni termo-igrometriche perfette. 

La teoria che si cela dietro l'attributo \emph{termo-igrometrico} è alla base della climatizzazione degli edifici.

Cosa si intende per \emph{climatizzazione}? Un edificio si definisce climatizzato o un impianto è di climatizzazione quando vengono controllati questi valori ambientali:
\begin{itemize}
	\item Temperatura;
	\item Umidità;
	\item Qualità dell'aria mediante un suo ricambio.
\end{itemize}
Il controllo di questi tre valori permette di vivere, come già è stato detto, un ambiente in condizioni termo-igrometriche ottimali. E solo un impianto di climatizzazione (solitamente un UTA -- \emph{Unità di Trattamento dell'Aria}) propriamente detto permette di ottenere questi risultati. Se, per esempio, non viene garantito il ricambio dell'aria (ovvero non viene immessa in ambiente una certa portata di aria esterna), l'impianto non è di climatizzazione.

La domanda successiva a cui bisogna dare una risposta è la seguente: \emph{quali sono le condizioni termo-igrometriche ottimali?}

La risposta è complicata anche se alcuni progettisti termo-tecnici tenderebbero subito a rispondere: \n{25}{\degreeCelsius} col \n{40}{\%} in estate e \n{20}{\degreeCelsius} col \n{50}{\%} in inverno.

In verità, per rispondere concretamente a questa domanda bisogna introdurre la teoria che è alla base della climatizzazione degli edifici. E per farlo si parte dal soggetto del primo periodo di questo capitolo: \emph{l'essere umano}.

L'uomo, in quanto essere vivente, trasforma continuamente l'energia chimica presente negli alimenti in forme più consone al mantenimento in vita del proprio corpo e alla sua trasformazione: quest'attività prende il nome di \emph{metabolismo}.

Questa continua trasformazione produce calore che risulta variabile in base all'attività svolta: stare seduto, camminare, correre, fare attività pesante e così via.

Non è difficile concludere che, considerando l'essere umano come un sistema chiuso, questo calore prodotto produce un accrescimento della temperatura corporea. Idealmente fino all'infinito. Ovviamente, l'essere umano non è un sistema chiuso e questo calore interno deve essere smaltito in relazione alle condizioni climatiche dell'ambiente esterno e, contemporaneamente, deve garantire la temperatura interna corporea.

In termini matematici il tutto si traduce in questo bilancio di potenze
\begin{equation}
	\label{bilanciocorpoumano}
	S=M-W-E_d-E_{sw}-E_{ve}-C_{ve}-C-R-C_k
\end{equation}
dove:
\begin{description}
	\item[$S$]Variazione di energia del corpo umano nell'unità di tempo o accumulo di energia termica nell'unità di tempo;
	\item[$M$]Potenza metabolica;
	\item[$E_d$] Potenza termica dispersa per diffusione attraverso la pelle;
	\item[$E_{sw}$] Potenza termica dispersa per sudorazione attraverso la pelle;
	\item[$E_{ve}$] Potenza termica dispersa nella respirazione come "calore latente";
	\item[$C_{ve}$] Potenza termica dispersa nella respirazione come "calore sensibile";
	\item[$C$] Potenza termica dispersa per convezione;
	\item[$R$] Potenza termica dispersa per irraggiamento;
	\item[$C_k$] Potenza dispersa per conduzione;
\end{description}
È risaputo che nel caso in cui la temperatura esterna sia eccessivamente alta o bassa, il meccanismo termoregolatorio del corpo tenda rispettivamente a far sudare il corpo stesso o a provocare i brividi. Nel caso in cui questi due ultimi meccanismi non siano sufficienti a mantenere costante l'energia interna del corpo si ha l'\emph{ipertermia} (fino alla morte per danni reversibili alle proteine dei tessuti nervosi) o l'\emph{ipotermia} (fino alla morte per fibrillazione cardiaca).

Quindi, è molto importante vivere in un ambiente che abbia delle condizioni termo-igrometriche tali da non provocare la morte prematura del proprio corpo.

Il vero problema è che queste così agognate \emph{condizioni termo-igrometriche ideali} non sono universali. E il motivo risiede nel fatto che ogni persona è diversa dalle altre. Infatti, ognuno tende a vestirsi diversamente (ovvero cambia la resistenza termica che il corpo oppone verso l'esterno e contemporaneamente anche il suo fattore di vista nel caso dell'irraggiamento) ma sopratutto ognuno tende ad avere un'attività metabolica $M$ completamente diversa. È come se ogni persona desiderasse una propria temperatura. Ovviamente è praticamente impossibile realizzare una cosa del genere in un ambiente affollato.

Citando da \emph{Impianti di climatizzazione per l'edilizia}:
\begin{quote}
	Perché ci sia comfort termico, una condizione necessaria è che l'energia interna del corpo umano non aumenti nè diminuisca, ovvero che sia nullo il termine di accumulo che nella \ref{bilanciocorpoumano} è indicato come $S$; per $S=0$ questa equazione diventa una relazione del tipo:
	\begin{equation}
	\label{bilanciocorpoumanoSnullo}
		f(abbigliamento, attivit\grave{a}, t_a, v_a, \Phi, t_r, t_{sk}, E_{sw})=0
	\end{equation}
	che lega tra loro otto variabili: due legate al soggetto (abbigliamento e attività), quattro ambientali (temperatura $t_a$, velocità $v_a$ e umidità $\Phi$ dell'aria e temperatura media radiante $t_r$ -- ovvero temperatura di un ambiente fittizio termicamente uniforme che scambierebbe con l'uomo la stessa potenza termica radiante scambiata nell'ambiente reale) e due fisiologiche (temperatura della pelle $t_{sk}$ e potenza termica dispersa per sudorazione o percentuale di pelle bagnata dal sudore $E_{sw}$).
	
	In verità le due variabili fisiologiche non sono variabili independenti, ma dipendono con legge complessa dalle altre; 
	
	\sdots
	
	Secondo Fanger, perché siano verificate le condizioni di benessere, devono essere anche soddisfatte le due equazioni:
	\begin{gather}
	\label{Esw}
		E_{sw}=0.42A_b[(M-W)/A_b-58.2]\\
	\label{tsk}
		t_{sk}=35.7-0.0275(M-W)/A_b
	\end{gather}
	cioè i valori di $E_{sw}$ e di $t_{sk}$ reali in condizioni di comfort termico sono quelli che si ottengono dalle due ora scritte in funzione dell'attività realmente svolta dal soggetto.
	
	In definitiva le possibili condizioni di benessere termico sono le combinazioni delle sei variabili indipendenti che soddisfano contemporaneamente le equazioni \ref{bilanciocorpoumanoSnullo}, \ref{Esw} e la \ref{tsk}.
\end{quote}



\vspace{2cm}
Il calcolo del carico termico nella stagione estiva tiene conto non solo dei valori ambientali esterni di temperatura e umidità ma anche - e sopratutto aggiungerei - della radiazione solare.

La radiazione solare rappresenta un carico termico non indifferente e di cui bisogna tenere debitamente conto. Anzi, al giorno d'oggi, è quasi più importante progettare \underline{la struttura} di un edificio per "resistere" alla stagione estiva che non a quella invernale. Facendo un semplice ragionamento, infatti, è possibile capire come proprio la radiazione solare rappresenti un punto cruciale per la progettazione, appunto, dell'involucro stesso.

Considerando un edificio sito a Napoli, dalla \norvent si hanno questi valori di temperatura esterna:
\begin{itemize}
	\item \n{2}{\degreeCelsius} per la stagione invernale;
	\item \n{32}{\degreeCelsius} per la stagione estiva;
\end{itemize}
All'interno dell'edificio, invece, non si devono superare i \n{20}{\degreeCelsius} nella stagione invernale mentre in estate un buon livello di benessere si ottiene con \n{25}{\degreeCelsius}.

È facile notare che, a parità di involucro (e quindi di trasmittanza termica), il salto termico è maggiore in inverno (\n{18}{\degreeCelsius}) che in estate (\n{7}{\degreeCelsius}). Da ciò potremmo concludere che l'involucro debba essere di tipo resistivo (bassa trasmittanza) per resistere, appunto, al forte gradiente termico che si instaura in inverno. 

La realtà delle cose è ben diversa. Questo modo di procedere va molto bene nei Paesi del Nord Europa dove gli inverni sono molto rigidi (si scende tranquillamente sotto gli \n{0}{\degreeCelsius}) mentre le temperature estive sono confortevoli.

Alle nostre latitudini (\ang{41}), invece, l'estate è rappresentata da temperature molto più elevate ma sopratutto da una radiazione solare incidente notevole: si raggiungono i \n{700}{W/m^2}. Il tutto si traduce in un aumento di temperatura delle superfici esposte al sole. Questo fenomeno non è per niente trascurabile. Infatti, nella valutazione del carico termico estivo si usa solitamente la \emph{temperatura sole-aria} (\si{t^'_e}): nell'ipotesi che il solo scambio esterno fosse per sola convezione, la suddetta temperatura è quella che genera la stessa potenza termica che nella realtà viene scambiata anche per irraggiamento.

Per rendere l'idea di quanto l'irraggiamento giochi un ruolo cruciale in estate, questa \si{t^'_e} è pari a \n{61.6}{\degreeCelsius} il 21 Luglio (16:00) alle latitudini napoletane su una superficie esposta a Ovest quando alla stessa ora la temperatura esterna è circa \n{34.4}{\degreeCelsius}. 

Il problema vero e proprio nasce con i componenti trasparenti che permettono, come è risaputo, alla radiazione solare di entrare all'interno degli edifici.

In questo caso si agisce, oltre che sulla trasmittanza globale dell'infisso, anche trattando superficialmente il vetro dell'infisso stesso in modo tale da ottenere superfici che permettano di far entrare all'interno dell'edificio solo una parte della radiazione solare.

Famosi sono i vetri \emph{bassoemissivi} che risultano opachi alla radiazione infrarossa che le attraversa dall'interno verso l'esterno. Questo permette, durante la stagione invernale, di non far uscire verso l'esterno la radiazione infrarossa e, quindi, di risparmiare sul riscaldamento. 

Sempre per la stagione invernale, sono stati studiati i cosiddetti vetri a \emph{guadagno solare} che si fanno facilmente attraversare dalla radiazione solare. 

Esistono poi i vetri a \emph{controllo solare} che si dividono in: \emph{selettivi}, \emph{riflettenti} e \emph{assorbenti}. Siccome la prestazione di una superficie qualunque che scambia calore per irraggiamento è riassumibile nei tre valori
\begin{center}
	$\alpha + \tau + \rho = 1$
\end{center}
i tre vetri vengono trattati superficialmente in modo tale da far variare di volta in volta i tre coefficienti e ottenere, quindi, prestazioni differenti.

Per esempio, un vetro \emph{selettivo} avrà un elevato valore di $\tau$ in corrispondenza delle onde visibili della radiazione solare per poi annullarsi in corrispondenza del vicino infrarosso.

Il vetro \emph{riflettente} avrà, come già il nome suggerisce, un elevato valore della riflessività nel visibile: ad occhio nudo si comportano come degli specchi.

Infine, il vetro \emph{assorbente} ha un elevato valore di $\alpha$ nel visibile: ciò si traduce con un vetro apparentemente scuro. La problematica di questo trattamento risiede nel fatto che assorbendo la radiazione solare, il vetro tende a riscaldarsi in poco tempo e quindi trasmettere verso l'interno un'aliquota della radiazione solare incidente sulle onde dell'infrarosso: immaginando una parete vetrata di questa tipologia, in poco tempo è come avere una parete radiante che annulla totalmente i benefici di un trattamento per il \emph{controllo solare} estivo. Quindi, un vetro di questo genere viene sempre accoppiato con uno basso-emissivo in modo tale che la radiazione emessa dalla lastra assorbente viene bloccata da quella basso-emissiva. Ovviamente una configurazione di questo genere viene montata in modo tale che il vetro a controllo solare sia posto esternamente.

Da tutto ciò discendono queste conclusioni (o modi di progettare l'edificio):
\begin{itemize}
	\item In inverno è molto importante avere una trasmittanza bassa per limitare lo scambio termico con l'esterno (sia per i componenti opachi che trasparenti). I componenti trasparenti, inoltre, per migliorare il guadagno solare (ovvero permettere alla radiazione solare di entrare negli edifici diminuendo il carico termico da abbattere con gli impianti) dovrebbero venir posizionati in maniera rilevante sopratutto sulle superfici esposte a sud;
	\item In estate i componenti opachi dovrebbero avere una elevata \emph{trasmittanza termica periodica} in modo tale che l'onda di calore dovuta alla radiazione solare incidente la parete venga attenuta e sfasata opportunamente. I componenti trasparenti dovrebbero essere limitati, se non addirittura assenti, sulle superfici esposte a est e ovest. A sud il problema dell'ingresso di radiazione solare tramite il vetro a guadagno solare viene risolto posizionando esternamente tendine o coperture: il sole in estate risulta, come è ben noto, più alto sulla volta celeste mentre è più basso in inverno. 
	\item A nord il componente trasparente non deve essere presente o, al limite, lo si sceglie del tipo \emph{basso-emissivo} per contenere le dispersioni di radiazione infrarossa. 
\end{itemize}