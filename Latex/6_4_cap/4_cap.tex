\chapter{Lo stato di progetto}
\thispagestyle{empty}
Nei capitoli precedenti è stata esposta la situazione attuale civile e meccanica. Per quanto riguarda la prima, in vista di una riqualificazione energetica, è ovvio che tutti i componenti debbano essere ``trattati'' in qualche modo per rispettare i limiti imposti dal Decreto Ministeriale del 26 Giugno 2015. Verrà, poi, effettuato un rifacimento di tutta la parte meccanica per rispettare il DPR del 14 Gennaio 1997 che si rifà al DL del 30 Dicembre 1992 in cui vengono, appunto, definiti i \emph{requisiti strutturali tecnologici e organizzativi minimi richiesti per l'esercizio delle attività sanitarie da parte delle strutture sanitarie pubbliche e private}.

La descrizione delle scelte effettuate partirà dall'involucro per poi giungere alla parte impiantistica prima lato utenza e poi lato sotto-centrale.

\section{L'involucro}
Si mostrano prima i requisiti da rispettare, tratti dal DM 26/6/2015, per i componenti opachi e trasparenti.
\begin{itemize}
\item Valori limite per la trasmittanza strutture opache verticali verso l'esterno -- DM 26/6/2015
\begin{center}
	\begin{tabular}{ccc}
	\multirow{2}{*}{Zona Climatica} & \multicolumn{2}{c}{\textbf{U} [\trasm]}	\\
	\cmidrule(lr){2-3}
	& \textbf{2015} & \textbf{2021}				\\
	\midrule
	A e B							&	0.45		&	0.40 					\\
	C								& 	0.40		&	0.36					\\
	D								&	0.36		&	0.32					\\
	E								&	0.30		&	0.28					\\
	F								&	0.28		&	0.26					\\
\end{tabular}
\end{center}

\newpage
\item Valori limite per la trasmittanza delle strutture opache orizzontali o inclinate di copertura, verso l'esterno -- DM 26/6/2015
\begin{center}
	\begin{tabular}{ccc}
		\multirow{2}{*}{Zona Climatica} & \multicolumn{2}{c}{\textbf{U} [\trasm]}	\\
		\cmidrule(lr){2-3}
		& \textbf{2015} & \textbf{2021}				\\
		\midrule
		A e B							&	0.34		&	0.32					\\
		C								& 	0.34		&	0.32					\\
		D								&	0.28		&	0.26					\\
		E								&	0.26		&	0.24					\\
		F								&	0.24		&	0.22					\\
	\end{tabular}
\end{center}
\item Valori limite per la trasmittanza delle chiusure trasparenti e opache, verso l'esterno -- DM 26/6/2015
\begin{center}
	\begin{tabular}{ccc}
		\multirow{2}{*}{Zona Climatica} & \multicolumn{2}{c}{\textbf{U} [\trasm]}	\\
		\cmidrule(lr){2-3}
		& \textbf{2015} & \textbf{2021}				\\
		\midrule
		A e B							&	3.20		&	3.00					\\
		C								& 	2.40		&	2.00					\\
		D								&	2.10		&	1.80					\\
		E								&	1.90		&	1.40					\\
		F								&	1.70		&	1.00					\\
	\end{tabular}
\end{center}
\end{itemize}
Nel secondo capitolo è già stato spiegato l'importanza di costruire componenti opachi che siano in grado di limitare le dispersione in inverno, seguendo le tabelle sopra riportate, e attenuare, oltre che sfasare, le rientrate estive seguendo le indicazioni sotto riportate del Decreto Ministeriale 26/6/15.
\begin{quote}
	Ad esclusione della zona F per le località in cui il valore medio mensile dell'irradianza sul piano orizzontale nel mese di massima insolazione $I_{m,s} \ge \SI{290}{W/m^2}$, verificare che:
	\begin{itemize}
		\item per le pareti opache verticali (ad eccezione di quelle nel quadrante Nord-Ovest/Nord/Nord-Est) sia rispettata almeno una delle seguenti condizioni:
		\begin{itemize}
			\item $M_s > \SI{260}{kg/m^2}$
			\item $Y_{IE}<\SI{0.10}{W/m^2K}$
		\end{itemize}
		\item per tutte le pareti opache orizzontali e inclinate, che:
		\begin{itemize}
			\item $Y_{IE}<\SI{0.18}{W/m^2K}$
		\end{itemize}
	\end{itemize}
	dove:\\
	$M_s$: rappresenta la massa superficiale della parete opaca compresa la malta dei giunti ed esclusi gli intonaci [\si{kg/m^2}];\\
	$Y_{IE}$: rappresenta la trasmittanza termica periodica valutata in accordo con UNI EN ISO 13786:2008 e successivi aggiornamenti [\si{W/m^2K}].
	
	Note:
	\begin{itemize}
		\item Gli effetti positivi che si ottengono con il rispetto dei valori di massa superficiale o trasmittanza termica periodica delle pareti opache, possono essere raggiunti, in alternativa, con l'utilizzo di tecniche e materiali, anche innovativi, ovvero coperture a verde, che permettano di contenere le oscillazioni della temperatura degli ambienti in funzione dell'irraggiamento solare. \sdots
	\end{itemize}
\end{quote}

Per quanto riguarda la parte opaca, si è proceduto ad effettuare un cappotto interno: quello esterno è escluso in quanto non è possibile far variare l'aspetto dell'edificio stesso. 

Il cappotto ha molteplici benefici:
\begin{itemize}
	\item modificando la stratigrafia, si abbatte la trasmittanza termica fino a \n{0.34}{\trasm};
	\item si aumenta la trasmittanza termica periodica alternando sapientemente materiali resistivi e capacitivi;
	\item si abbattono notevolmente i ponti termici con minori rischi di condensa.
\end{itemize}
L'intervento verrà effettuato sui seguenti componenti opachi: \textbf{Muro EXT}, \textbf{Muro EXT 200} e \textbf{Muro EXT Corpo Basso}. I materiali scelti, come è già stato detto, sono di tipo resistivo e capacitivo.
\begin{itemize}
	\item Le fibre di PET da riciclo hanno una conduttività pari a $\lambda=\SI{0.034}{W/mK}$ e una densità $\rho=\SI{50}{kg/m^3}$ permettendogli di avere un carattere decisamente resistivo;
	\item il \emph{CELENIT N-C} avendo una conduttività più elevata ($\lambda=\SI{0.065}{W/mK}$), a cui si contrappone una densità pari a $\rho=\SI{400}{kg/m^3}$ un calore specifico di $c_s=\SI{1810}{kJ/kgK}$, svolge la funzione capacitiva nell'involucro.
\end{itemize}
Il modo più intelligente di posizionare gli strati di Celenit e fibre di PET è quello di ``difendere'' il materiale più capacitivo dalle alterazioni termiche esterne con quello più resistivo. Posando il Celenit esternamente, infatti, questo tenderà a riscaldarsi maggiormente nelle giornate estive rilasciando il calore durante la notte: quindi anche se la trasmittanza $U$ della parete risulta uguale nelle due configurazioni, il comportamento dinamico della stessa risulta essere peggiore. Al contrario, le fibre di PET diminuiscono il carico a cui è sottoposto il Celenit che avrà, quindi, una temperatura più consona al benessere interno.

La stratigrafia del cappotto interno è in comune a tutti e 3 i componenti opachi:
\begin{itemize}
	\item \n{5}{cm} di fibre di PET da riciclo;
	\item \n{5}{cm} di CELENIT;
	\item \n{1.25}{cm} di lastre di cartongesso ($\lambda=\SI{0.21}{W/mK}$ -- $\rho=\SI{900}{kg/m^3}$);
	\item \n{1.5}{cm} di PVC come rifinitura ($\lambda=\SI{0.17}{W/mK}$ -- $\rho=\SI{1390}{kg/m^3}$)
\end{itemize}

In questo modo la trasmittanza termica $U$ risulta essere inferiore ai limiti imposti dal decreto, la trasmittanza termica periodica $Y_{IE}$ è inferiore al limite dello \n{0.10}{\trasm} (valori di attenuazione pari a circa \num{0.046} mentre lo sfasamento si attesta sulle \num{19} ore). Le lastre di cartongesso servono a contenere i due pannelli isolanti mentre il PVC svolge la funzione di rifinitura.

La copertura del Corpo A non risulta essere oggetto di intervento; quella del Corpo C, invece, viene coibentata esternamente con \emph{poliuretano a spruzzo}: $\lambda=\SI{0.24}{W/mK}$ -- $\rho=\SI{60}{kg/m^3}$. Per arrivare ai limiti imposti si deve formare uno strato di \n{7}{cm}: $U=\SI{0.30}{\trasm}$ e $Y_{IE}=\SI{0.063}{\trasm}$.

I componenti trasparenti vengono sostituiti integralmente cercando di mantenere inalterata l'aspetto esterno. 

Le caratteristiche comuni di tutti gli infissi sono:
\begin{itemize}
	\item Telaio in alluminio anodizzato con taglio termico ($U_f=\SI{1.5}{\trasm}$);
	\item Stratigrafia del vetro-camera ($U_g=\SI{1.57}{\trasm}$):
		\begin{itemize}
			\item Vetro Interno: $s=\SI{4}{mm}$;
			\item Intercapedine: $s=\SI{20}{mm}$ con Argon e tendine interne;
			\item Vetro Esterno: $s=\SI{6}{mm}$ --- $\epsilon=0.2$;
		\end{itemize}
\end{itemize}
che si riassumono in un valore di trasmittanza pari a $U_w \approx \SI{1.85}{\trasm}$.

Il comportamento invernale dei componenti trasparenti viene migliorato tramite l'utilizzo di un vetro basso-emissivo che limita la fuoriuscita di radiazione infrarossa generata dai corpi caldi presenti all'interno dell'edificio; il comportamento estivo del serramento viene migliorato utilizzando le tendine poste all'interno dell'intercapedine per evitare l'entrata di radiazione solare sopratutto all'alba e al tramonto quando il sole è più o meno perpendicolare all'involucro. Non si sono preferiti vetri a controllo solare perché la facciata meridionale è più esposta a SUD che a EST.

Questi dati sull'involucro sono stati inseriti nel programma \textbf{CYPETHERM Load} che ha restituito i nuovi valori dei carichi termici. Con queste nuove potenze termiche verrà dimensionato l'impianto e la sua sotto-centrale.
\section{I Risultati Energetici}
\subsection{La Stagione Estiva}
Le zone termiche \utic\ e \emod\ hanno visto diminuire il proprio carico termico (sia estivo che invernale). Il \blocc, invece, che non viene interessato dall'intervento, ha restituito gli stessi valori. Mantenendo la ventilazione costante, i risultati sono riassunti nella seguente tabella che mostra il carico termico estivo.
\begin{center}
	\begin{tabular}{lccccc}
		\toprule
		&	Superficie 				&	Portata Aria Est. 			&	$\dot{Q}_{sol}+\dot{Q}_{trasm}$		& 	\multirow{2}{*}{RST}		&	$\dot{Q}_{frigo}$ 	\\
		&	{\small \si{m^2}}		&		{\small \si{kg/s}}		&		{\small \si{kW}}				&								&{\small \si{kW}}		\\					
		\midrule	
		UTIC		&		\num{146}			&		\num{1.46}				&	\num{3.54}		&	\num{0.91}					&	\num{6.39}		\\
		Emod.		&		\num{169}			&		\num{1.69}				&	\num{3.78}		&	\num{0.90}					&	\num{7.08}		\\
		B.O.		&		\num{697}			&		\num{7.33}				&	\num{34.6}		&	\num{0.94}					&	\num{48.3}		\\
		\bottomrule
	\end{tabular}
\end{center}
Paragonando i risultati con lo stato di fatto, si nota una netta diminuzione soprattutto per l'Emodinamica. Questa, infatti, ha un involucro costituito da molti serramenti. L'involucro trasparente dell'UTIC, invece, è coperto internamente da una parete: quindi si percepirà meno l'effetto benefico dei nuovi infissi.

Anche in questo caso, come nello stato di fatto, conoscendo il valore di RST, della portata di ventilazione e del carico da bilanciare si ottengono le condizioni termo-igrometriche dell'aria da immettere. Siccome il carico latente rimane costante ma diminuisce quello sensibile, la RST per UTIC e Emodinamica tende a diminuire ma comunque di una quantità trascurabile. Un risultato da aspettarsi, invece, è l'aumento della temperatura di immissione dell'aria in ambiente: a portata d'aria costante, diminuendo il carico da abbattere, l'aria da immettere sarà sicuramente più calda. Lo si evince tranquillamente da questo bilancio:
\begin{align}
\dot{m}h_s + \dot{Q}_{trasm}	&	=\dot{m}h_i											\label{bilancioEST}\\
h_s								&	=h_i-\frac{\dot{Q}_{trasm}}{\dot{m}}				\notag\\
t_s								&	=\frac{t_ic_p\dot{m}-\dot{Q}_{trasm}}{c_p\dot{m}}	\label{bilancioEST2}
\end{align} 
Nella \vref{bilancioEST2} vale la stessa ipotesi di aria secca.
Conoscendo quindi le condizioni ambientali esterne e quelle a cui bisogna immettere l'aria nei locali, sono state calcolate le potenze termiche necessarie, in \si{kW}. Si noti l'aumento di potenza da fornire alla batteria di post-riscaldamento.
\begin{center}
	\begin{tabular}{lcc}
		&	$Q_{UTA,f}$		&	$Q_{UTA,c}$\\
		\midrule
		UTIC	&	\num{79.3}			&	\num{11.1}\\
		Emod.	&	\num{91.7}			&	\num{13.2}\\
		B.O.	&	\num{664}			&	\num{39.0}\\
	\end{tabular}
\end{center}
Per quanto riguarda le altre zone termiche i carichi sensibili e latenti da abbattere sono i seguenti.
\begin{center}
	\begin{tabular}{lcccc}
		\toprule
		\multirow{2}*{Zona Termica} & Superficie 			& $Q_L$ 			& $Q_S$ 				& $\dot{Q}_{frigo}$  \\
									& \si{m^2}				& \si{kW}			& \si{kW}				& \si{kW}\\
		\midrule
		Radiatori		& \num{291.1}						& ---				& ---					& ---\\
		Corpo C			& \num{529.2}						& \num{2.3}			& \num{52.0}			& \num{54.3}\\
		Corpo A			& \num{2352.2}						& \num{20.6}		& \num{102}				& \num{123}\\
		\bottomrule
	\end{tabular}
\end{center}
Si può ben vedere come il carico sensibile si è ridotto di quasi la metà per entrambe le zone termiche (\corpa\ e \textbf{C}).

I suddetti carichi termici non tengono conto della ventilazione. Per valutarla è necessario considerare la tipologia di impianto che si vuole installare. Per garantire una buona qualità del comfort termo-igrometrico si è preferito utilizzare un impianto \emph{misto} con aria primaria, per il rinnovo dell'aria e l'abbattimento del calore latente, e fancoil idronici per il calore sensibile. In questa sede non si vogliono dimensionare le unità locali ma valutare solo i carichi termici in gioco i quali verranno utilizzati per effettuare il vero e proprio dimensionamento.

La portata di ventilazione necessaria per garantire una buona qualità dell'aria è presente nella \norvent\ per ogni destinazione d'uso. Sommandole si ottiene la portata che deve essere trattata dall'eventuale UTA:
\begin{itemize}
	\item Per il \corpa, la portata di ventilazione è pari a \n{15254}{m^3/h}. Tenendo conto di un fattore di sicurezza del \n{20}{\%}, il valore effettivamente usato in fase progettuale è \n{18303}{m^3/h};
	\item Per il \corpc, la portata di ventilazione è pari a \n{8777}{m^3/h}. Tenendo conto del fattore di sicurezza, il valore utilizzato è \n{10532}{m^3/h}.
\end{itemize}
Con questi valori si valuta il carico termico aggiuntivo necessario per portare l'aria esterna nelle condizioni neutre interne: \emph{neutre} perchè suddetta portata ha la sola funzione di rinnovare l'aria interna e di abbattere il carico latente. La tabella che segue riporta tutti i valori dei carichi termici necessari per dimensionare l'impianto per la stagione estiva.
\begin{center}
	\small
	\begin{tabular}{lccccc}
		\toprule
		\multirow{2}*{Zona Termica} & Superficie 						& $\dot{Q}_L$ 			& $\dot{Q}_S$ 		& $\dot{Q}_{vent}$		& $\dot{Q}_{frigo}$  \\
									& \si{m^2}							& \si{kW}			& \si{kW}		&\si{kW}			& \si{kW}\\
		\midrule
		UTIC						& \num{146}							& \num{0.583}		& \num{5.80}	& \num{68.3}		& \num{74.7} \\
		Emo.						& \num{291.1}						& \num{0.674}		& \num{7.08}	& \num{78.7}		& \num{86.5}\\
		B.O.						& \num{291.1}						& \num{2.79}		& \num{45.5}	& \num{360}			& \num{408}\\
		Radiatori					& \num{291.1}						& ---				& ---			& ---				& ---\\
		Corpo C						& \num{529.2}						& \num{2.3}			& \num{52.0}	& \num{287}			& \num{341}\\
		Corpo A						& \num{2352.2}						& \num{20.6}		& \num{102}		& \num{165}			& \num{288}\\
		\bottomrule
	\end{tabular}
\end{center}
Dove:
\begin{equation}
	\dot{Q}_{frigo}=\dot{Q}_L+\dot{Q}_S+\dot{Q}_{vent}
\end{equation}
Facendo un paragone tra questa tabella e quella \vpageref{stato:fatto}, si può notare come, sopratutto per i due corpi A e C, ci sia una netta diminuzione dei carichi termici dovuti all'involucro, come è già stato detto, ma il carico totale $\dot{Q}_{frigo}$ aumenta a causa della ventilazione: nel Corpo C sono presenti molti laboratori che necessitano di un tasso di rinnovo dell'aria pari a \n{6}{vol/h} mentre il Corpo A è costituito sostanzialmente da degenze e studi medici. La \norvent\ prevede per questi locali o \n{50}{m^3/h} per persona o, nel caso in cui non sia noto il numero di persone che solitamente affollano il locale, si assicurano i \n{2}{vol/h}: per sicurezza si utilizza il valore maggiore tra i due risultati.

\subsection{La Stagione Invernale}
Partendo dal solito trittico (\utic, \emod\ e \blocc) si ottengono questi risultati:
\begin{center}
	\label{UTA:potenzaPROGETTO}
	\begin{tabular}{lcccc}
		\toprule
		\multirow{2}{*}{Zona Termica}				& \multirow{2}{*}{RST}	&	$t_s$ 					& $\dot{Q}_{trasm}$	&	$\dot{Q}_{vent}$	\\
							&						&	\si{\degreeCelsius} &	\si{kW}				&	\si{kW}\\
		\midrule
		UTIC			&\multirow{3}{*}{1.1}	&	\num{23.6}	&\num{2.29}		&	\num{52.7}\\
		Emod.			&						&	\num{25.8}	&\num{6.49}		&	\num{66.1}\\
		B.O.			&						&	\num{27.2}	&\num{38.4}		&	\num{293}\\
		\bottomrule
	\end{tabular}
\end{center}
È ovvio che le temperature di immissione dell'aria nei locali diminuiscano in quanto sono diminuiti i carichi termici. Per il Blocco Operatorio la situazione non cambia come era prevedibile. 

Per quanto riguarda le altre zone termiche i risultati da aspettarsi saranno una diminuzione della potenza $\dot{Q}_{trasm}$ mentre, a causa della ventilazione, si avrà un notevole aumento della $\dot{Q}_{risc}$. I risultati di tutti i carichi invernali, nello stato di progetto, sono riassunti nella seguente tabella.
\begin{center}
	\begin{tabular}{lccccc}
		\toprule
		\multirow{2}*{Zona Termica} & Superficie 		&  $\dot{Q}_{trasm}$	& Ventilazione		&  	 $\dot{Q}_{vent}$			& $\dot{Q}_{risc}$		\\
		& [\si{m^2}]				& \si{kW}			& \si{kW}			&	\si{m^3/h}		& \si{kW}		\\
		\midrule
		Radiatori					& \num{291.1}		& \num{12.3}		& --			& ---						& \num{12.3}	\\
		B.O.						& \num{696.7}		& \num{38.4}		& \num{21936}	& \num{293}					& \num{331}	\\
		UTIC						& \num{145.7}		& \num{2.29}		& \num{4371}	& \num{52.7}				& \num{55.0}	\\
		Emod.						& \num{164.4}		& \num{6.49}		& \num{5058}	& \num{66.1}				& \num{72.6} 	\\
		Corpo C						& \num{529.2}		& \num{24.3}		& \num{8777}	& \num{123}					& \num{147}	\\
		Corpo A						& \num{2352.2}		& \num{45.8}		& \num{15254}	& \num{200}					& \num{246}	\\
		\bottomrule
	\end{tabular}
\end{center}




\vspace{2cm}
Inserisci qua le migliorie da realizzare sull'edificio 2 del policlinico. Dividi le migliorie per tipologia:
\begin{itemize}
	\item componenti opachi
	\item componenti finestrati
	\item impianto
\end{itemize}
Descrivi poi i risultati ottenuti dividendo sempre per i vari casi studio. Alla fine metti una tabella riassuntiva con i risultati ottenuti, con le differenza tra stato attuale e di progetto divisi sempre per caso. Inserisci i costi da sostenere per realizzare il progetto (eventuale).

Inserisci uno schema (al limite sempre in formato A3 da piegare nella tesi) dell'impianto (sia sottocentrale termofrigorifera che aeraulica) spiegandone il funzionamento.