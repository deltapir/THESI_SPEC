\chapter{Lo stato di progetto}
\thispagestyle{empty}
Nei capitoli precedenti è stata esposta la situazione attuale civile e meccanica. Per quanto riguarda la prima, in vista di una riqualificazione energetica, è ovvio che tutti i componenti debbano essere ``trattati'' in qualche modo per rispettare i limiti imposti dal Decreto Ministeriale del 26 Giugno 2015. Verrà, poi, effettuato un rifacimento di tutta la parte meccanica per rispettare il DPR del 14 Gennaio 1997 che si rifà al DL del 30 Dicembre 1992 in cui vengono, appunto, definiti i \emph{requisiti strutturali tecnologici e organizzativi minimmi richiesti per l'esercizio delle attività sanitarie da parte delle strutture sanitarie pubbliche e private}.

La descrizione delle scelte effettuate partirà dall'involucro per poi giungere alla parte impiantistica prima lato utenza e poi lato sotto-centrale.

\section{L'involucro}
Si mostrano prima i requisiti da rispettare, tratti dal DM26/06/2015, per i componenti opachi e trasparenti.
\begin{itemize}
\item Valori limite per la trasmittanza strutture opache verticali verso l'esterno -- DM 26/6/2015
\begin{center}
	\begin{tabular}{ccc}
	\multirow{2}{*}{Zona Climatica} & \multicolumn{2}{c}{\textbf{U} [\trasm]}	\\
	\cmidrule(lr){2-3}
	& \textbf{2015} & \textbf{2021}				\\
	\midrule
	A e B							&	0.45		&	0.40 					\\
	C								& 	0.40		&	0.36					\\
	D								&	0.36		&	0.32					\\
	E								&	0.30		&	0.28					\\
	F								&	0.28		&	0.26					\\
\end{tabular}
\end{center}

\newpage
\item Valori limite per la trasmittanza delle strutture opache orizzontali o inclinate di copertura, verso l'esterno -- DM 26/6/2015
\begin{center}
	\begin{tabular}{ccc}
		\multirow{2}{*}{Zona Climatica} & \multicolumn{2}{c}{\textbf{U} [\trasm]}	\\
		\cmidrule(lr){2-3}
		& \textbf{2015} & \textbf{2021}				\\
		\midrule
		A e B							&	0.34		&	0.32					\\
		C								& 	0.34		&	0.32					\\
		D								&	0.28		&	0.26					\\
		E								&	0.26		&	0.24					\\
		F								&	0.24		&	0.22					\\
	\end{tabular}
\end{center}
\item Valori limite per la trasmittanza delle chiusure trasparenti e opache, verso l'esterno -- DM 26/6/2015
\begin{center}
	\begin{tabular}{ccc}
		\multirow{2}{*}{Zona Climatica} & \multicolumn{2}{c}{\textbf{U} [\trasm]}	\\
		\cmidrule(lr){2-3}
		& \textbf{2015} & \textbf{2021}				\\
		\midrule
		A e B							&	3.20		&	3.00					\\
		C								& 	2.40		&	2.00					\\
		D								&	2.10		&	1.80					\\
		E								&	1.90		&	1.40					\\
		F								&	1.70		&	1.00					\\
	\end{tabular}
\end{center}
\end{itemize}
Nel secondo capitolo è già stato spiegato l'importanza di costruire componenti opachi che siano in grado di limitare le dispersione in inverno, seguendo le tabelle sopra riportate, e attenuare, oltre che sfasare, le rientrate estive seguendo le indicazioni sotto riportate del Decreto Ministeriale 26/6/15.
\begin{quote}
	Ad esclusione della zona F per le località in cui il valore medio mensile dell'irradianza sul piano orizzontale nel mese di massima insolazione $I_{m,s} \ge \SI{290}{W/m^2}$, verificare che:
	\begin{itemize}
		\item per le pareti opache verticali (ad eccezione di quelle nel quadrante Nord-Ovest/Nord/Nord-Est) sia rispettata almeno una delle seguenti condizioni:
		\begin{itemize}
			\item $M_s > \SI{260}{kg/m^2}$
			\item $Y_{IE}<\SI{0.10}{W/m^2K}$
		\end{itemize}
		\item per tutte le pareti opache orizzontali e inclinate, che:
		\begin{itemize}
			\item $Y_{IE}<\SI{0.18}{W/m^2K}$
		\end{itemize}
	\end{itemize}
	dove:\\
	$M_s$: rappresenta la massa superficiale della parete opaca compresa la malta dei giunti ed esclusi gli intonaci [\si{kg/m^2}];\\
	$Y_{IE}$: rappresenta la trasmittanza termica periodica valutata in accordo con UNI EN ISO 13786:2008 e successivi aggiornamenti [\si{W/m^2K}].
	
	Note:
	\begin{itemize}
		\item Gli effetti positivi che si ottengono con il rispetto dei valori di massa superficiale o trasmittanza termica periodica delle pareti opache, possono essere raggiunti, in alternativa, con l'utilizzo di tecniche e materiali, anche innovativi, ovvero coperture a verde, che permettano di contenere le oscillazioni della temperatura degli ambienti in funzione dell'irraggiamento solare. \sdots
	\end{itemize}
\end{quote}

Per quanto riguarda la parte opaca, si è proceduto ad effettuare un cappotto interno: quello esterno è escluso in quanto non è possibile far variare l'aspetto dell'edificio stesso. 

Il cappotto ha molteplici benefici:
\begin{itemize}
	\item si abbatte la trasmittanza termica fino a \n{0.34}{\trasm};
	\item si aumenta la trasmittanza termica periodica alternando sapientemente materiali resistivi e capacitivi;
	\item si abbattono notevolmente i ponti termici con minori rischi di condensa.
\end{itemize}
L'intervento verrà effettuato sui seguenti componenti opachi: \textbf{Muro EXT}, \textbf{Muro EXT 200} e \textbf{Muro EXT Corpo Basso}. I materiali scelti, come è già stato detto, sono di tipo resistivo e capacitivo.
\begin{itemize}
	\item Le fibre di PET da riciclo hanno una conduttività pari a $\lambda=\SI{0.034}{W/mK}$ e una densità $\rho=\SI{50}{kg/m^3}$ permettendogli di avere un carattere decisamente resistivo;
	\item il \emph{CELENIT N-C} avendo una conduttività più elevata ($\lambda=\SI{0.065}{W/mK}$), a cui si contrappone una densità pari a $\rho=\SI{400}{kg/m^3}$, svolge la funzione capacitiva nell'involucro.
\end{itemize}
La posizione di questi materiali è stata fatta privilegiando il comportamento estivo, ovvero posizionando il materiale più resistivo all'esterno per limitare la potenza termica entrante in fase estiva e quindi evitare che una massa volumica maggiore (come il CELENIT) possa riscaldarsi.

La stratigrafia del cappotto interno è in comune a tutti e 3 i componenti opachi:
\begin{itemize}
	\item \n{5}{cm} di fibre di PET da riciclo;
	\item \n{5}{cm} di CELENIT;
	\item \n{1.25}{cm} di lastre di cartongesso ($\lambda=\SI{0.21}{W/mK}$ -- $\rho=\SI{900}{kg/m^3}$);
	\item \n{1.5}{cm} di PVC come rifinitura ($\lambda=\SI{0.17}{W/mK}$ -- $\rho=\SI{1390}{kg/m^3}$)
\end{itemize}

In questo modo la trasmittanza termica $U$ risulta essere inferiore ai limiti imposti dal decreto, la trasmittanza termica periodica $Y_{IE}$ è inferiore al limite dello \n{0.10}{\trasm} (valori di attenuazione pari a circa \num{0.046} mentre lo sfasamento si attesta sulle \num{19} ore)

La copertura del Corpo A non risulta essere oggetto di intervento; quella del Corpo C, invece, viene coibentata esternamente con \emph{poliuretano a spruzzo}: $\lambda=\SI{0.24}{W/mK}$ -- $\rho=\SI{60}{kg/m^3}$. Per arrivare ai limiti imposti si deve formare uno strato di \n{7}{cm}: $U=\SI{0.30}{\trasm}$ e $Y_{IE}=\SI{0.063}{\trasm}$.

I componenti trasparenti vengono sostituiti integralmente cercando di mantenere inalterata l'aspetto esterno. 

Le caratteristiche comuni di tutti gli infissi saranno:
\begin{itemize}
	\item Telaio in alluminio anodizzato con taglio termico ($U_f=\SI{1.5}{\trasm}$);
	\item Stratigrafia del vetro-camera ($U_g=\SI{1.57}{\trasm}$):
		\begin{itemize}
			\item Vetro Interno: $s=\SI{4}{mm}$;
			\item Intercapedine: $s=\SI{20}{mm}$ con Argon e tendine interne;
			\item Vetro Esterno: $s=\SI{6}{mm}$ --- $\epsilon=0.2$;
		\end{itemize}
\end{itemize}
che si riassumono in un valore di trasmittanza pari a $U_w \approx \SI{1.85}{\trasm}$.

Il comportamento invernale dei componenti trasparenti viene migliorato tramite l'utilizzo di un vetro basso-emissivo che limita la fuoriuscita di radiazione infrarossa generata dai corpi caldi presenti all'interno dell'edificio; il comportamento estivo del serramento viene migliorato utilizzando le tendine poste all'interno dell'intercapedine per evitare l'entrata di radiazione solare sopratutto all'alba e al tramonto quando il sole è più o meno perpendicolare all'involucro. Non si sono preferiti vetri a controllo solare perché la facciata meridionale è più esposta a SUD che a EST. Stesso discorso per la facciata settentrionale.

\vspace{2cm}
Inserisci qua le migliorie da realizzare sull'edificio 2 del policlinico. Dividi le migliorie per tipologia:
\begin{itemize}
	\item componenti opachi
	\item componenti finestrati
	\item impianto
\end{itemize}
Descrivi poi i risultati ottenuti dividendo sempre per i vari casi studio. Alla fine metti una tabella riassuntiva con i risultati ottenuti, con le differenza tra stato attuale e di progetto divisi sempre per caso. Inserisci i costi da sostenere per realizzare il progetto (eventuale).

Inserisci uno schema (al limite sempre in formato A3 da piegare nella tesi) dell'impianto (sia sottocentrale termofrigorifera che aeraulica) spiegandone il funzionamento.