\thispagestyle{empty}
Grazie
%%\Huge Grazie\dots}\\
%\begin{flushright}
%\Large\textsf{È fastidioso ringraziare solo in queste occasioni,\\bisognerebbe farlo più spesso.}	
%\end{flushright}
%
%I primi "Grazie" vanno ai miei genitori e alle mie sorelle che mi hanno sopportato per 26 anni (e continueranno a farlo\dots) e supportato soprattutto in questo ultimo periodo. Vi ringrazio anche (o soprattutto?) per avermi ``inculcato'' l'amore per la curiosità che mi ha portato ad intraprendere la strada più azzeccata che abbia mai fatto in vita mia: ottenere un pezzo di carta con su scritto che sono ingegnere.
%
%Ringrazio gli ingegneri che mi hanno formato e permesso di arrivare in questo punto della mia carriera.\\
%Il \tit{Prof.}{Adolfo Palombo} per aver convogliato quel marasma di idee green sull'efficientamento energetico in qualcosa di effettivamente utile;
%ringrazio Cesare Forzano e Annamaria Buonomano per la vostra gentilezza innanzitutto e poi per l'aiuto resomi disponibile alla fine (?) di questo percorso;
%ringrazio gli ingegneri Cristian Monfrecola e Lucio Pandolfi per avermi sottoposto ad una terapia forzata e necessaria di nonnismo!
%Ringrazio l'\tit{Ing.}{Gioacchino Forzano} per la sua umiltà e inspiegabilmente eccessiva disponibilità.\\
%Ringrazio la mia seconda famiglia: Ada, Nicola, Mario, Marco, Guido, Giovanni, Stefania, Eleonora, Silvia, Davide e Carlo. O dovrei ringraziare qualcuno ai piani alti per averci fatto incontrare?\\
%A Roma e Pisa, invece, saluto il gruppo ``\emph{Ke sfaccim la Kastla, il KKK, Wagliu e altre creature leggendarie}'' (ovvero i compagni Materazzo, Bucciarelli e Antonelli). \\
%Ringrazio i miei amici Scout che considero da tempi immemori il mio rifugio tranquillo e disordinato dal rumore del mondo: Sara, Federica, Edoardo, Susanna, Chiara, Jasmine, Bruno\dots\\
%Ringrazio il gruppo dell'università per la stima che ci unisce, per il vostro sempre presente parere scientifico e per le burle: Biagio, Antonio, Gennaro, Pasquale, Giuseppe, Saverio e Luigi. \\
%Ringrazio, abbraccio e saluto i miei cugini: Annamaria e Nicoletta, Luca e Gabriella, Luigi e Filomena, Francesco e Riccardo, Erika e Giovanni. Ringrazio tutto il resto della famiglia (nonni, zii, parenti vicini e lontani) perché siamo uniti e vicini in ogni occasione anche se ci separano chilometri (e in alcuni casi \emph{migliaia} di chilometri).\\
%Saluto e ringrazio i miei professori del Liceo Scientifico "B. Pascal" di Pomezia per avermi mostrato, dimostrato e trasmesso nitidamente e a colori l'amore per la cultura: \tit{Prof.ssa}{Cadelli}, \tit{Prof.}{Rossi}, \tit{Prof.ssa}{Nardecchia}, \tit{Prof.}{Russo} e la \tit{Prof.ssa}{Zadra}.
%
%\vspace{0.5em}
%Ognuno di voi mi ha dato qualcosa a proprio modo e per questo vi ringrazio. Non sono un paroliere quindi sembra che ogni parola (o combinazione di loro) non riesca a esprimere al meglio quello che voi siete per me. Sappiate che ognuno di voi è per me una spalla su cui piangere in caso di bisogno, una stanza vuota in cui sfogarmi, un panorama in cui rilassarsi, una compagnia per una passeggiata e una squadra (un po' scarsa!) di calcetto, siete una tavolata imbandita a pasquetta e un pieno di diesel per ogni viaggio, siete un bicchiere di vino per quando bisogna scherzare e un silenzio comprensivo, siete un caldo abbraccio e la corda giusta da pizzicare in un concerto. 
%
%Molto probabilmente ho dimenticato qualcuno: nel dubbio ti ringrazio, perchè in qualche modo (o nel bene o nel male) mi hai dato qualcosa. D'alto canto se stai leggendo vuol dire che in qualche modo ti sei interessato/a: e quindi grazie!
%
%Spero solo che almeno per un istante da quando vi conosco sia riuscito in qualche modo a ricambiare quanto mi avete dato.