\chapter{La prima costruzione del Policlinico}
\thispagestyle{empty}
%\addcontentsline{toc}{section}{L'involucro}
%Si riportano stralci dei primi capitoli del III Volume \textsc{Ospedali e Cliniche Universitarie} dell'Ing. \textsc{Corrado Beguinot}. In questo modo, si spera di riuscire a evidenziare quelle che sono le differenze progettuali tra le normative vigenti all'epoca della costruzione del Policlinico e quelle odierne.\vspace{0.5em}
%Tutta questa parte scrivila con i margini più grandi, ovvero il testo deve essere più schiacciato orizzontalmente.
\begin{quoting}
	\emph{Con l'inizio dell'anno accademico 1972-73 la seconda Facoltà di Medicina e Chirurgia dell'\uni{} ha iniziato la sua attività nella sua nuova sede, all'epoca in parte ultimata ed in parte in corso di ultimazione.}
	
	\emph{Il numero degli studenti supera oggi le \num{6000} unità ed il numero complessivo dei letti è di \num{2758}.}

\emph{La Facoltà è costituita da un organismo edilizio articolato in ventisei edifici nei quali hanno sede gli Istituti, le Cliniche, i servizi e le attrezzature che sono collegati da gallerie di servizio a due livelli e da una viabilità principale e secondaria, e dotati di ampie superfici a verde e di parcheggi. La superficie complessiva su cui è stata realizzata la Facoltà è di \n{440000}{m^2} ed il volume costruito è di \n{11130000}{m^3} con una superficie totale dei piani pari a \n{270000}{m^2}.}

\emph{Il costo dell'opera, comprensivo del costo dei suoli, delle attrezzature didattiche, dell'arredo, delle sistemazioni esterne, degli impianti e degli oneri revisionali, è risultato di circa \num{45} miliardi con un costo quindi a \si{m^3} di \num{40000} lire circa.}

\emph{Gli istituti sono inseriti nei vari edifici in funzione delle loro affinità didattiche e di ricerca ed in funzione della prevista organizzazione dipartimentale.}

\end{quoting}
\noindent Le prime righe di questo elaborato di laurea sono state volutamente lasciate alle parole dell'\tit{ing.}{Corrado Beguinot} il quale coordinò i lavori di progettazione e costruzione del policlinico. 

Questo elaborato di laurea ha come obiettivo lo studio dell'Edificio 2 dell'AOU \uni{} (allora \emph{Patologia Medica}, ora \emph{Cardiochirurgia}) in vista di una sua riqualificazione energetica: sia per quanto riguarda l'involucro che gli impianti annessi. Volendo usare altre parole e riassumere molto banalmente il tutto, lo scopo di questo studio è quello di aumentare la classe energetica dell'edificio stesso. Il raggiungimento di questo obiettivo è iniziato con uno studio delle condizioni attuali dell'edificio. Sono stati effettuati vari sopralluoghi ma sopratutto si è cercato di recuperare materiale disponibile in letteratura. Sono risultati molto utili il terzo volume scritto dall'ingegnere Corrado Beguinot, di cui sono riportati i passi più utili, e l'analisi del \tit{prof.}{Adolfo Palombo} nel Giugno 2013 sugli edifici 9A, 9F e 9H. In questo modo è stato possibile ricavare dati preziosi riguardanti l'edificio 2 senza dover ricorrere ad analisi invasive delle pareti o coperture che comportano spese sicuramente economiche ma anche temporali.

Volendo entrare più nello specifico, si riportano gli stralci più importanti del suddetto terzo volume che hanno permesso di semplificare lo studio iniziale dell'edificio di patologia medica dell'AOU \uni. 

I primi passi, riportati all'inizio di questo capitolo, sono una spiegazione generica di come è stato concepito l'intero complesso ospedaliero. Andando avanti l'ingegnere si concentra sui corpi alti descrivendo la stratigrafia delle pareti verticali esterne e interne, gli infissi esterni (ovvero l'involucro trasparente) e quelli interni, per poi passare alle coperture. Seguono gli edifici bassi e i tunnel di collegamento. Dopo aver finito con la parte relativa all'architettura e all'edilizia si concentra sugli impianti utilizzati.\\
Inerentemente all'edificio 2, il Beguinot si esprime così:
\begin{quoting}
	L'edificio di \emph{Patologia Medica} comprende\emph{: laboratori di ricerca e stabulari per una superficie di circa \n{1200}{m^2}; 1 aula da \num{400} posti, \num{8} aule da \num{30} in comune con la Semeiotica Medica, l'Endocrinologia e la Dermatologia; reparti di degenza per un totale di \num{150} posti-letto, con una superficie di circa \n{4400}{m^2}. La superficie totale dei piani è di \n{8700}{m^2} circa.}
\end{quoting}
Per quanto riguarda l'involucro opaco e trasparente, invece:
\begin{quoting}
	\emph{L'adozione di un sistema modulare generale ha consentito una impostazione unitaria della costruzione dei complessi delle Cliniche, costituiti dai corpi alti delle degenze e dai corpi delle piastre di base degli ambulatori, delle diagnostiche, degli uffici, dei laboratori, delle aule, ecc.}
	
	\emph{La struttura dei corpi di degenza è essenzialmente costituita da una teoria perimetrale di pilastri pressoinflessi, posti ad interasse di \n{1.90}{m}, collegati ai piani da un solaio c.a. realizzato con coppie di travi a sezione rettangolare, agganciate lateralmente ai pilastri, e solette piene in calcestruzzo. La rigidezza globale del sistema è assicurata da una trave perimetrale parapetto e dai blocchi dei collegamenti verticali, scala principale, scale di servizio e di emergenza, costituiti da pareti portanti in c.a. dello spessore di \n{40}{cm}, dalle quali escono a sbalzo i gradini prefabbricati ed i ripiani. Il coronamento dell'edificio è realizzato con elementi modulari prefabbricati in c.a., agganciati al prolungamento dei pilastri.}

\sdots

	\emph{Le facciate dei corpi degenza sono caratterizzate, oltre che dagli elementi pieni di calcestruzzo armato dei collegamenti verticali, da una alternanza di pieni e vuoti, realizzati con l'adozione di un elemento tipo prefabbricato, di lunghezza modulare, ancorato mediante piastre, perni e bulloni ai pilastri. I suddetti blocchi sono costituiti da un doppio strato di silicalcite pesante e leggera, e sono rifiniti sulla faccia esterna con una superficie granigliata e sul lato interno a stucco. Lo spazio libero, tra la fascia parapetto e l'intradosso del solaio, è modulato in sette parti uguali, che sono occupate da elementi prefabbricati o elementi di infisso in accordo con le esigenze di visibilità e di illuminazione degli ambienti interni e della composizione della superficie esterna.}

\sdots

	\emph{Allo scopo di attenuare la rumorosità negli ambienti sono stati adottati nei corridoi e nelle altre zone di traffico (atrio al piano, soggiorno pazienti, testate di servizio) pavimenti di gomma; nelle camere di degenza, nei soggiorni per i visitatori, e negli studi si è adottata una pavimentazione resiliente. Sia i pavimenti di gomma che quelli resilienti sono posti in opera su sottofondo di arena e cemento. In tutti i locali di servizio, in quelli soggetti a più frequenti lavaggi e disinfezioni, cioè ambienti per la visita medica, laboratori, ecc., si è adottata una pavimentazione in grés opaco.}

\sdots

	\emph{Per l'alloggiamento degli impianti sono stati predisposti nelle strutture appositi cavedi, ubicati in posizione tali da servire in maniera omogenea la superficie di ogni piano. Più precisamente sono stati realizzati due grandi cavedi in corrispondenza rispettivamente della scala di sicurezza all'estremità del corpo di fabbrica e del torrino dei servizi, che è anche attraversato verticalmente da altri tre cavedi più piccoli, dei quali due per impianti idrico-sanitari ed uno di forma allungata, riservato esclusivamente agli impianti elettrici. Una serie continua di cavedi attraversa longitudinalmente tutto il corpo della degenza, accogliendo le condotte pluviali, gli scarichi degli impianti idrici annessi alle camere di degenza, le canne di aspirazione delle cappe dei laboratori, nei casi in cui è prevista l'ubicazione di questi ultimi nel piano rialzato del blocco degenza, e le reti di distribuzione dei gas medicali.}

\sdots

	\emph{I cavedi, grandi e piccoli, ed ogni altra canalizzazione verticale, al piano cantinato, fanno capo alle reti orizzontali, ospitate nella galleria di servizio.}
	
	\emph{L'isolamento termico delle coperture è ottenuto con l'interposizione, tra solaio ed impermeabilizzazione, di uno strato coibente, sul quale è disposto un masso concreto con pendenza verso l'interno del corpo di fabbrica per il raccordo delle pluviali, sistemate nei cavedi precedentemente descritti.}

\sdots

	\emph{Gli infissi esterni sono realizzati con profilo in lamierino di acciaio verniciato a fuoco e vetro semidoppio; quelli delle camere di degenza, alternati con i blocchi di silicalcite, hanno apertura a vasistas e tenda di oscuramento in tessuto plastico; quelli del torrino dei servizi e dell'atrio di piano realizzano una fascia continua verticale, interrotta solo dallo spessore dei solai e dalla fascia parapetto, ed hanno apertura a battente.}

\sdots

	\emph{Per quanto riguarda la finitura delle superfici interne ed esterne si ricorda che sono lasciati a vista i getti di calcestruzzo sia all'interno che all'esterno, proteggendoli con vernice idrorepellente ed antipolvere. Le altre pareti ed i soffitti sono dipinti con pittura lavabile. In particolare quelle delle camere di degenza, realizzati, come già detto con solette e travi binate, hanno il calcestruzzo a vista per le travi e l'intradosso delle solette, gettate su cassaforma di compensato marino, semplicemente stuccate e dipinte con pittura lavabile.}
\end{quoting}
\noindent I corpi bassi differiscono di poco dalla struttura modulare di quelli alti. Infatti:
\begin{quoting}
	\emph{La struttura portante dei corpi bassi conserva la stessa modulazione dei blocchi degenze.}

\sdots

	\emph{La trave parete, oltre alla funzione di collegamento e di irrigidimento dell'insieme, ha anche quella di contenimento del terreno per il piano di servizio che corre continuo alle piastre.}
	
	\emph{Le coperture di tali corpi sono realizzate con elementi in cemento armato ad U capovolto, sul modulo tipo di \n{1.90}{m}, poggiati su travi portanti longitudinali, ed uscenti a sbalzo per \n{1.60}{m}, onde realizzare una zona d'ombra sulle pareti verticali.}

\sdots

	\emph{Uno strato coibente ed un'impermeabilizzazione a guaina sintetica proteggono le coperture, seguendone il profilo grecato.}

\sdots

	\emph{Le facciate esterne sono costituite dall'alternanza di elementi di tompagno e di infissi, suggerita dalla funzionalità esterna.}
	
	\emph{I tompagni sono realizzati con doppia fodera: pannello prefabbricato di cemento granigliato all'esterno, e tavolato di mattoni forati intonacato all'interno. Gli infissi sono in lamiera di acciaio verniciato a fuoco di tipo analogo a quelli del corpo di fabbrica delle degenze.}
\end{quoting}

%\addcontentsline{toc}{section}{L'impianto}
\noindent Per quanto riguarda la parte impiantistica si vogliono riportare ancora stralci de \emph{Ospedali e Cliniche universitarie} scritto dall'\tit{Ing.}{Corrado Beguinot}: si tratta in questo caso del secondo capitolo che ha come titolo ‘‘\emph{Gli Impianti Termofrigoriferi~-~\emph{Centrale Termofrigorifera, Sottocentrali ed Impianti Interni}}''.
\begin{quoting}
	\textbf{Centrale Termofrigorifera} -- \emph{La Centrale Termofrigorifera, ubicata nell'area delle attrezzature centralizzata, è stata realizzata dalla Marelli Aerotecnica} \dots
		
\sdots 
	
	\emph{L'edificio della Centrale occupa un'area di \n{2850}{m^2} con un volume entrofuoriterra pari a \n{21000}{m^3}. È realizzato con strutture in cemento armato a faccia a vista e con tecnologie analoghe a quelle prevalenti nel complesso Policlinico e, in particolare, per gli impianti tecnologici centralizzati.}
	
	\emph{L'opera realizzata dalla Marelli afferma il principio della centralizzazione della produzione del caldo e del freddo per tutte le utenze relative ai molteplici Istituti e Servizi. I vantaggi del sistema adottato possono condensarsi nei seguenti punti: minor potenzialità installata, riserva a costi minori, impiego di combustibili densi ed eventuale utilizzazione futura del gas metano, concentrazione stoccaggio combustibili con minori costi, maggiore durata con rendimenti più alti, minori investimenti con minori costi di manutenzione, minori consumi d'acqua con minori costi di produzione, riduzione delle fonti di inquinamento.}
	
	\emph{Nella Centrale Termofrigorifera sono state installate: 5 caldaie con una potenzialità complessiva di circa \n{60d6}{kcal/h}, 5 gruppi frigoriferi, di cui uno del tipo ad assorbimento avente una capacità di \n{3d6}{frig/h}\footnote{Si ricorda che la \emph{frigoria} è l'opposto di una \emph{caloria}. Pertanto \n{d6}{frig/h} equivalgono a circa \n{116.3}{kW}} e 4 di tipo Centrifugo con capacità singola di \n{6d6}{frig/h}. Le caldaie forniscono vapore surriscaldato, sia per la produzione di acqua surriscaldata delle sottocentrali, quanto per l'azionamento delle turbine collegate ai gruppi frigoriferi centrifughi e delle turbopompe di circolazione dei fluidi prodotti dall'impianto frigorifero (acqua refrigerata e acqua raffreddamento condensatori).}
	
	\emph{Le caldaie sono del tipo pressurizzato e collegate ognuna ad un recuperatore per il massimo sfruttamento del calore sensibile dei prodotti della combustione.}
	
	\emph{Ogni caldaia produce vapore alla pressione di \n{40}{kg/cm^2}\footnote{Circa \n{40}{bar}} surriscaldato a circa \n{360}{\degreeCelsius}.}
	
	\emph{L'energia termica, sia per il periodo invernale come per il periodo estivo, per il riscaldamento e produzione acqua calda necessaria ai servizi, viene distribuita mediante acqua surriscaldata ad una temperatura di circa \n{170}{\degreeCelsius}, non oltre i \n{180}{\degreeCelsius}.}
	
	\sdots
	
	\emph{La circolazione dell'acqua surriscaldata è assicurata da più pompe elettrocentrifughe che, attraverso un'estesa rete di distribuzone, servono ogni fabbisogno termico attraverso scambiatori installati nelle varie sottostazioni dislocate negli scantinati dei divers fabbricati.}
	
	\emph{La Centrale Termica è integrata da un impianto di trattamento di acqua grezza, sia per il riempimento del sistema acqua surriscaldata, sia per il reintegro spurghi caldaie.}
	
	\emph{Affiancata alla Centrale Termica, ma divisa, in rispetto delle norme VV.FF., è stata realizzata la Centrale Frigorifera con una serie di 4 gruppi centrifughi in parallelo ed uno ad assorbimento; i primi, azionati da turbine in contropressione, mentre l'assorbitore utilizza, in parte, il vapore di scarico delle turbopompe di circolazione dell'acqua di condensazione refrigerata.}
	
	\emph{L'adozione della macchina ad assorbimento è stata attuata per potere sfruttare meglio il vapore e poter far funzionare gli impianti primari e secondari in modo soddisfacente anche a bassi carichi continui (funzionamenti notturni, periodo invernale), in modo che possa essere tenuto in funzione il maggior numero di ore/anno in relazione alle fluttuazioni dei carichi dal 100\% alle minime percentuali con i più elevati rendimenti, e con semplice regolazione.}
	
	\emph{Il vapore ad alta pressione ed alta temperatura, proveniente dalle caldaie, viene distribuito in parallelo: alle quattro turbine a condensazione, accoppiate direttamente ai quattro compressori centrifughi, ai quattro turboriduttori a contropressione delle pompe di circolazione dell'acqua di raffreddamento dei condensatori e circolazione dell'acqua refrigerata.}
	
	\sdots
	
	\emph{Data la notevole quantità d'acqua necessaria al raffreddamento dei condensatori dei gruppi centrifughi, dell'assorbitore e condensatori dell'utilizzo vapore al fine di limitare i consumi, l'acqua di raffreddamento è inviata a un complesso di torri di raffreddamento che la riporta ad una temperatura di utilizzo per uno sfruttamento in ciclo chiuso. La torre, che nel suo assieme è costituita da otto torri con otto bacini di raccolta acqua, realizzati in cemento armato, ha una lunghezza di 50 metri, larga 10 metri ed oltre 15 metri in altezza; tratta \n{6000}{m^3/h} di acqua per una potenzialità di scambio pari ad oltre 45 milioni di \si{cal/h}.}
\end{quoting}
Alla fine del capitolo viene riassunto il funzionamento di ogni sottocentrale presente nello scantinato di ogni corpo di fabbrica.
\begin{quoting}
	\textbf{Sottocentrali} - \emph{Tutte le sottocentrali sono collegate alla rete primaria dell'acqua surriscaldata la quale cede il proprio carico termico  al circuito secondario a mezzo di scambiatori di calore di tipo acqua-acqua.}
	
	\emph{I due regimi idraulici dei due fluidi, del primario e del secondario, sono così indipendenti.}
	
	\emph{L'eventuale esclusione del circuito secondario e la sua parzializzazione è permessa by-passando l'acqua attraverso una valvola a tre vie, rimanendo così costante la portata d'acqua nella rete primaria.}
	
	\emph{L'acqua surriscaldata soddisfa pure i fabbisogni termici per i servizi igienici alimentando i collettori; sono previsti tutti gli accorgimenti onde garantire la sicurezza del servizio.}
	
	\emph{Per l'acqua refrigerata invece il circuito primario è collegato direttamente alle reti secondarie attraverso valvole miscelatrici; pertanto i due circuiti, primario e secondario, sono collegati idraulicamente.}

\sdots


\noindent\emph{Le sottocentrali e gli impianti interni degli istituti clinici comprendono:}	
\begin{enumerate}
	\item Sottostazioni di produzione del calore;
	\item Sottostazioni di produzione del freddo;
	\item Quadri elettrici e regolazioni automatiche;
	\item Distribuzione del calore e del freddo;
	\item Condizionamento dell'aria;
	\item Riscaldamento a piastre;
	\item Estrazione aria viziata servizi degenze;
	\item Produzione e distribuzione dell'aria compressa;
	\item Produzione e distribuzione del vuoto;
	\item Distribuzione del gas illuminante;
\end{enumerate}
\subsubsection{Sottostazione di produzione del calore}
\emph{Le sottostazioni realizzate sono:}
\begin{itemize}
	\item Medicina Generale
	\item Semeiotica Medica
	\item Patologia Medica
	\item Medicina del Lavoro
	\item Ostetrica e Ginecologica
	\item Radiologia
	\item Pediatrica e Puericultura
	\item Malattie Nervose e Mentali
	\item Malattie Infettive
	\item Ortopedica
	\item Otorino
	\item Odontoiatrica
	\item Neurochirugica
	\item Chirurgia Generale
\end{itemize}
\vspace{0.5em}
	\noindent\emph{In ciascuna delle predette sottostazioni sono state installate le apparecchiature di trasformazione del calore costituite, per ognuna, da due scambiatori per la produzione di acqua calda a \num{90}} + \emph{\num{95}\si{\degreeCelsius} e due scambiatori per la produzione dell'acqua calda a \n{60}{\degreeCelsius}.}
	
	\emph{Gli scambiatori sono alimentati dal fluido primario costituito da acqua surriscaldata proveniente dai distributori correnti nel cunicolo alla temperatura di circa \n{170}{\degreeCelsius} e con ritorno di \n{110}{\degreeCelsius}, con una pressione disponibile tra andata e ritorno di 10 metri di colonna d'acqua.}
	
	\emph{Ogni scambiatore è dotato di propria apparecchiatura di termoregolazione automatica, a funzionamento pneumatico, costituito essenzialmente da valvola deviatrice a tre vie, comandata da termostato posto sulla scia del secondario.}
	
	\emph{Sono state installate nelle stesse sottocentrali i collettori di andata e ritorno dell'acqua calda alle diverse temperature e le relative elettropompe di circolazione; sono stati inoltre installati i dispositivi necessari per mantenere costante la pressione dell'acqua surriscaldata in arrivo alle sottostazioni.}
	
	\emph{Tutte le apparecchiature delle sottostazioni sono state superdimensionate per tener conto di eventuali futuri fabbisogni termici.}
	\subsubsection{Sottostazioni di produzione del freddo}
	\emph{Negli stessi locali sono stati portati gli allacciamenti dell'acqua refrigerata di circa \n{4}{\degreeCelsius} in arrivo e ritorno a \n{12}{\degreeCelsius}. Per ciascuna sottostazione è stata prevista un'apparecchiatura regolatrice, stabilizzatrice di pressione, ed un contatore di frigorie con totalizzatore settimanale. È stata inoltre posta un'apparecchiatura di regolazione automatica atta a mantenere costante la temperatura di ritorno a \n{12}{\degreeCelsius}.}

	\sdots
	\subsubsection{Distribuzione del calore e del freddo}
	\emph{Dai collettori posti nelle sottostazioni partono le tubazioni di trasporto dell'acqua calda e fredda per l'alimentazione dei vari impianti di condizionamento e di riscaldamento.}
	
	\emph{Queste reti sono state realizzate in tubi di acciaio trafilato <<Mannesmann>>, posti in opera mediante saldature ossiacetileniche, verniciate antiruggine ed isolate termicamente; completano le reti le saracinesche d'intercettazione.}
	\subsubsection{Condizionamento dell'aria}
	\emph{Molteplici sono i tipi d'impianti realizzati nelle varie Cliniche e precisamente: per i locali adibiti ad uffici e laboratori sono stati realizzati impianti <<ad induzione>> con aria primaria atta ad assicurare un ricambio di circa \n{2}{Vol/h}.}
	
	\emph{Altro tipo di impianto è stato quello a doppio condotto ad alta velocità con cassette miscelatrici-attenuatrici e l'alimentazione delle bocchette a bassa velocità; ciascuna cassetta miscelatrice è comandata da termostato in ambiente. Gli ambienti sono condizionati a tutt'aria esterna ed i ricambi di aria sono stati determinati in funzione dei carichi termici e delle necessità di ventilazione dovuta ai particolari usi a cui i locali stessi sono stati adibiti.}

	\emph{Ciascun impianto di condizionamento si completa con l'impianto di estrazione dell'aria con smaltimento diretto all'esterno.}
	
	\emph{Gli stabulari ed i locali delle terapie sono stati trattati con impianti indipendenti del tipo convenzionale a tutt'aria esterna.}
	
	\emph{I ricambi assicurati sono pari ad almeno \n{12}{Vol/h}.}
	
	\sdots
	\subsubsection{Riscaldamento a piastre}
	\emph{Nei locali degenze e servizi sono stati realizzati impianti di riscaldamento con termoconvettori in prevalenza in alluminio pressofuso. In questi ambienti la temperatura garantita è di \n{20}{\degreeCelsius}, anche in relazione alla minima esterna invernale di \n{0}{\degreeCelsius}.}
	
	\sdots
	
	\emph{Ciascuna colonna montante o discendente dell'impianto di riscaldamento è dotata di saracinesche d'intercettazione in bronzo pesante con rubinetti di scarico a maschio, con attacco portagomma sul ritorno.}
	
	\emph{Per ogni corpo di fabbrica è stato previsto un circuito diverso intercettabile con saracinesche nella sottostazione.}
	\subsubsection{Estrazione aria viziata dei servizi degenze}
	\emph{Per questi impianti sono state installate bocchette di ripresa a portata regolabile in maniera da permettere di regolare perfettamente la portata di aria ripresa nel singolo ambiente. Esse fanno capo ad una canalizzazione verticale in lamiera di ferro zincato prolungata fino alla copertura e collegata ad un collettore di raccolta dell'aria da smaltire, facente capo ad un ventilatore di espulsione dell'aria viziata.}
\end{quoting}

Inserisci eventualmente foto prese dal Beguinot o foto dal sopralluogo

