\chapter{La prima costruzione del Policlinico}
\thispagestyle{empty}
\addcontentsline{toc}{section}{Il Progetto}
Si riportano stralci dei primi capitoli del III Volume \textsc{Ospedali e Cliniche Universitarie} dell'Ing. \textsc{Corrado Beguinot}. In questo modo, si spera di riuscire a evidenziare quelle che sono le differenze progettuali tra le normative vigenti all'epoca della costruzione del Policlinico e quelle odierne.\vspace{0.5em}
 
\emph{Con l'inizio dell'anno accademico 1972-73 la seconda Facoltà di Medicina e Chirurgia dell'\uni ha iniziato la sua attività nella sua nuova sede, all'epoca in parte ultimata ed in parte in corso di ultimazione.}

\emph{Il numero degli studenti supera oggi le \num{6000} unità ed il numero complessivo dei letti è di \num{2758}.}

\emph{La Facoltà è costituita da un organismo edilizio articolato in ventisei edifici nei quali hanno sede gli Istituti, le Cliniche, i servizi e le attrezzature che sono collegati da gallerie di servizio a due livelli e da una viabilità principale e secondaria, e dotati di ampie superfici a verde e di parcheggi. La superficie complessiva su cui è stata realizzata la Facoltà è di \n{440000}{m^2} ed il volume costruito è di \n{11130000}{m^3} con una superficie totale dei piani pari a \n{270000}{m^2}.}

\emph{Il costo dell'opera, comprensivo del costo dei suoli, delle attrezzature didattiche, dell'arredo, delle sistemazioni esterne, degli impianti e degli oneri revisionali, è risultato di circa \num{45} miliardi con un costo quindi a \si{m^3} di \num{40000} lire circa.}

\emph{Gli istituti sono inseriti nei vari edifici in funzione delle loro affinità didattiche e di ricerca ed in funzione della prevista organizzazione dipartimentale.}

\sdots

\noindent Inerentemente all'oggetto di questa tesi, il Beguinot descrive così l'edificio 2:

\vspace{0.5em}

L'edificio di \emph{Patologia Medica} comprende\emph{: laboratori di ricerca e stabulari per una superficie di circa \n{1200}{m^2}; 1 aula da \num{400} posti, \num{8} aule da \num{30} in comune con la Semeiotica Medica, l'Endocrinologia e la Dermatologia; reparti di degenza per un totale di \num{150} posti-letto, con una superficie di circa \n{4400}{m^2}. La superficie totale dei piani è di \n{8700}{m^2} circa.}

\vspace{0.5em}
\noindent In merito all'involucro opaco e trasparente, l'ingegnere si esprime così:

\vspace{0.5em}

\emph{L'adozione di un sistema modulare generale ha consentito una impostazione unitaria della costruzione dei complessi delle Cliniche, costituiti dai corpi alti delle degenze e dai corpi delle piastre di base degli ambulatori, delle diagnostiche, degli uffici, dei laboratori, delle aule, ecc.}

\emph{La struttura dei corpi di degenza è essenzialmente costituita da una teoria perimetrale di pilastri pressoinflessi, posti ad interasse di \n{1.90}{m}, collegati ai piani da un solaio c.a. realizzato con coppie di travi a sezione rettangolare, agganciate lateralmente ai pilastri, e solette piene in calcestruzzo. La rigidezza globale del sistema è assicurata da una trave perimetrale parapetto e dai blocchi dei collegamenti verticali, scala principale, scale di servizio e di emergenza, costituiti da pareti portanti in c.a. dello spessore di \n{40}{cm}, dalle quali escono a sbalzo i gradini prefabbricati ed i ripiani. Il coronamento dell'edificio è realizzato con elementi modulari prefabbricati in c.a., agganciati al prolungamento dei pilastri.}

\sdots

\emph{Le facciate dei corpi degenza sono caratterizzate, oltre che dagli elementi pieni di calcestruzzo armato dei collegamenti verticali, da una alternanza di pieni e vuoti, realizzati con l'adozione di un elemento tipo prefabbricato, di lunghezza modulare, ancorato mediante piastre, perni e bulloni ai pilastri. I suddetti blocchi sono costituiti da un doppio strato di silicalcite pesante e leggera, e sono rifiniti sulla faccia esterna con una superficie granigliata e sul lato interno a stucco. Lo spazio libero, tra la fascia parapetto e l'intradosso del solaio, è modulato in sette parti uguali, che sono occupate da elementi prefabbricati o elementi di infisso in accordo con le esigenze di visibilità e di illuminazione degli ambienti interni e della composizione della superficie esterna.}

\sdots

\emph{Allo scopo di attenuare la rumorosità negli ambienti sono stati adottati nei corridoi e nelle altre zone di traffico (atrio al piano, soggiorno pazienti, testate di servizio) pavimenti di gomma; nelle camere di degenza, nei soggiorni per i visitatori, e negli studi si è adottata una pavimentazione resiliente. Sia i pavimenti di gomma che quelli resilienti sono posti in opera su sottofondo di arena e cemento. In tutti i locali di servizio, in quelli soggetti a più frequenti lavaggi e disinfezioni, cioè ambienti per la visita medica, laboratori, ecc., si è adottata una pavimentazione in grés opaco.}

\sdots

\emph{Per l'alloggiamento degli impianti sono stati predisposti nelle strutture appositi cavedi, ubicati in posizione tali da servire in maniera omogenea la superficie di ogni piano. Più precisamente sono stati realizzati due grandi cavedi in corrispondenza rispettivamente della scala di sicurezza all'estremità del corpo di fabbrica e del torrino dei servizi, che è anche attraversato verticalmente da altri tre cavedi più piccoli, dei quali due per impianti idrico-sanitari ed uno di forma allungata, riservato esclusivamente agli impianti elettrici. Una serie continua di cavedi attraversa longitudinalmente tutto il corpo della degenza, accogliendo le condotte pluviali, gli scarichi degli impianti idrici annessi alle camere di degenza, le canne di aspirazione delle cappe dei laboratori, nei casi in cui è prevista l'ubicazione di questi ultimi nel piano rialzato del blocco degenza, e le reti di distribuzione dei gas medicali.}

\sdots

\emph{I cavedi, grandi e piccoli, ed ogni altra canalizzazione verticale, al piano cantinato, fanno capo alle reti orizzontali, ospitate nella galleria di servizio.}

\emph{L'isolamento termico delle coperture è ottenuto con l'interposizione, tra solaio ed impermeabilizzazione, di uno strato coibente, sul quale è disposto un masso concreto con pendenza verso l'interno del corpo di fabbrica per il raccordo delle pluviali, sistemate nei cavedi precedentemente descritti.}

\sdots

\emph{Gli infissi esterni sono realizzati con profilo in lamierino di acciaio verniciato a fuoco e vetro semidoppio; quelli delle camere di degenza, alternati con i blocchi di silicalcite, hanno apertura a vasistas e tenda di oscuramento in tessuto plastico; quelli del torrino dei servizi e dell'atrio di piano realizzano una fascia continua verticale, interrotta solo dallo spessore dei solai e dalla fascia parapetto, ed hanno apertura a battente.}

\sdots

\emph{Per quanto riguarda la finitura delle superfici interne ed esterne si ricorda che sono lasciati a vista i getti di calcestruzzo sia all'interno che all'esterno, proteggendoli con vernice idrorepellente ed antipolvere. Le altre pareti ed i soffitti sono dipinti con pittura lavabile. In particolare quelle delle camere di degenza, realizzati, come già detto con solette e travi binate, hanno il calcestruzzo a vista per le travi e l'intradosso delle solette, gettate su cassaforma di compensato marino, semplicemente stuccate e dipinte con pittura lavabile.}

\vspace{0.5em}
\noindent I corpi bassi differiscono di poco dalla struttura modulare di quelli alti. Infatti:

\vspace{0.5em}

\emph{La struttura portante dei corpi bassi conserva la stessa modulazione dei blocchi degenze.}

\sdots

\emph{La trave parete, oltre alla funzione di collegamento e di irrigidimento dell'insieme, ha anche quella di contenimento del terreno per il piano di servizio che corre continuo alle piastre.}

\emph{Le coperture di tali corpi sono realizzate con elementi in cemento armato ad U capovolto, sul modulo tipo di \n{1.90}{m}, poggiati su travi portanti longitudinali, ed uscenti a sbalzo per \n{1.60}{m}, onde realizzare una zona d'ombra sulle pareti verticali.}

\sdots

\emph{Uno strato coibente ed un'impermeabilizzazione a guaina sintetica proteggono le coperture, seguendone il profilo grecato.}

\sdots

\emph{Le facciate esterne sono costituite dall'alternanza di elementi di tompagno e di infissi, suggerita dalla funzionalità esterna.}

\emph{I tompagni sono realizzati con doppia fodera: pannello prefabbricato di cemento granigliato all'esterno, e tavolato di mattoni forati intonacato all'interno. Gli infissi sono in lamiera di acciaio verniciato a fuoco di tipo analogo a quelli del corpo di fabbrica delle degenze.}

Inserisci eventualmente foto prese dal Beguinot o foto dal sopraluogo

Impianto termico e centrale tecnologica a pagina 89


\vspace{1em}
\begin{flushright}
	\textbf{\textsc{Corrado Beguinot}}
\end{flushright}
\newpage
\subsection{aeaefaefaef}
\subsubsection{aefaefaefaefaefa}