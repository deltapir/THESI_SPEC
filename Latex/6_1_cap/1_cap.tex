\chapter{La prima costruzione del Policlinico}
\thispagestyle{empty}
\addcontentsline{toc}{section}{Il Progetto}
\emph{Con l'inizio dell'anno accademico 1972-73 la seconda Facoltà di Medicina e Chirurgia dell'\uni ha iniziato la sua attività nella sua nuova sede, all'epoca in parte ultimata ed in parte in corso di ultimazione.}

\emph{Il numero degli studenti supera oggi le \num{6000} unità ed il numero complessivo dei letti è di \num{2758}.}

\emph{La Facoltà è costituita da un organismo edilizio articolato in ventisei edifici nei quali hanno sede gli Istituti, le Cliniche, i servizi e le attrezzature che sono collegati da gallerie di servizio a due livelli e da una viabilità principale e secondaria, e dotati di ampie superfici a verde e di parcheggi. La superficie complessiva su cui è stata realizzata la Facoltà è di \n{440000}{m^2} ed il volume costruito è di \n{11130000}{m^3} con una superficie totale dei piani pari a \n{270000}{m^2}.}

\emph{Il costo dell'opera, comprensivo del costo dei suoli, delle attrezzature didattiche, dell'arredo, delle sistemazioni esterne, degli impianti e degli oneri revisionali, è risultato di circa \num{45} miliardi con un costo quindi a \si{m^3} di \num{40000} lire circa.}

\emph{Gli istituti sono inseriti nei vari edifici in funzione delle loro affinità didattiche e di ricerca ed in funzione della prevista organizzazione dipartimentale.} 
\vspace{2em}
\begin{flushright}
	\textbf{\textsc{Corrado Beguinot}}
\end{flushright}
\newpage
\subsection{aeaefaefaef}
\subsubsection{aefaefaefaefaefa}