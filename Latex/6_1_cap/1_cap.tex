\chapter{La prima costruzione del Policlinico}
\thispagestyle{empty}
\addcontentsline{toc}{section}{Il Progetto}
Si riportano stralci dei primi capitoli del III Volume \textsc{Ospedali e Cliniche Universitarie} dell'Ing. \textsc{Corrado Beguinot}. In questo modo, si spera di riuscire a evidenziare quelle che sono le differenze progettuali tra le normative vigenti all'epoca della costruzione del Policlinico e quelle odierne.\vspace{0.5em}
 
\emph{Con l'inizio dell'anno accademico 1972-73 la seconda Facoltà di Medicina e Chirurgia dell'\uni ha iniziato la sua attività nella sua nuova sede, all'epoca in parte ultimata ed in parte in corso di ultimazione.}

\emph{Il numero degli studenti supera oggi le \num{6000} unità ed il numero complessivo dei letti è di \num{2758}.}

\emph{La Facoltà è costituita da un organismo edilizio articolato in ventisei edifici nei quali hanno sede gli Istituti, le Cliniche, i servizi e le attrezzature che sono collegati da gallerie di servizio a due livelli e da una viabilità principale e secondaria, e dotati di ampie superfici a verde e di parcheggi. La superficie complessiva su cui è stata realizzata la Facoltà è di \n{440000}{m^2} ed il volume costruito è di \n{11130000}{m^3} con una superficie totale dei piani pari a \n{270000}{m^2}.}

\emph{Il costo dell'opera, comprensivo del costo dei suoli, delle attrezzature didattiche, dell'arredo, delle sistemazioni esterne, degli impianti e degli oneri revisionali, è risultato di circa \num{45} miliardi con un costo quindi a \si{m^3} di \num{40000} lire circa.}

\emph{Gli istituti sono inseriti nei vari edifici in funzione delle loro affinità didattiche e di ricerca ed in funzione della prevista organizzazione dipartimentale.}

\vspace{0.5em}

[\dots]

\vspace{0.5em}
\noindent Inerentemente all'oggetto di questa tesi, il Beguinot descrive così l'edificio 2:

\vspace{0.5em}

L'edificio di \emph{Patologia Medica} comprende\emph{: laboratori di ricerca e stabulari per una superficie di circa \n{1200}{m^2}; 1 aula da \num{400} posti, \num{8} aule da \num{30} in comune con la Semeiotica Medica, l'Endocrinologia e la Dermatologia; reparti di degenza per un totale di \num{150} posti-letto, con una superficie di circa \n{4400}{m^2}. La superficie totale dei piani è di \n{8700}{m^2} circa.}

\vspace{0.5em}
\noindent In merito all'involucro opaco e trasparente, l'ingegnere si esprime così:

\vspace{0.5em}

L'adozione di un [vedi pagina 46\dots]

Inserisci eventualmente foto prese dal Beguinot o foto dal sopraluogo

Impianto termico e centrale tecnologica


\vspace{1em}
\begin{flushright}
	\textbf{\textsc{Corrado Beguinot}}
\end{flushright}
\newpage
\subsection{aeaefaefaef}
\subsubsection{aefaefaefaefaefa}