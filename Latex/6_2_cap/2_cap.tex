\chapter{Lo stato attuale}
\thispagestyle{empty}
\section{Le modifiche al giorno d'oggi}
%Aggiungi un mini indice in questo capitolo se le dimensioni dello stesso iniziano ad essere proibitive.\\
%Parla anche dell'assenza di organi per la misurazione dei consumi (elettrici e termici) e quindi problemi per la valutazione energetica \emph{ante-operam} e per il rispetto ai requisiti \emph{CAM}.\vspace{1em}
Nel primo capitolo si sono descritti in maniera sommaria l'architettura, l'edilizia e gli impianti presenti nell'intero complesso ospedaliero (al momento della costruzione) riportando le parole dell'\tit{ing.}{Corrado Beguinot}. 

In questo capitolo, invece, si vuole dare ampio spazio alle condizioni attuali del policlinico e dell'edificio stesso, riportando i dati di input inseriti all'interno dello studio pre-riqualificazione energetica riferiti, quindi, allo stato di fatto. 

In questi anni, infatti, nella centrale termica, le caldaie vengono fatte funzionare per inviare acqua calda nella rete di teleriscaldamento non più a \n{170}{\degreeCelsius} ma a \n{130}{\degreeCelsius}. Per quanto riguarda il teleriscaldamento, invece, il cogeneratore è stato modificato. Sono stati aggiunti questi gruppi frigoriferi di cui tot ad assorbimento.

Bisogna precisare che l'assorbitore non riesce a coprire il carico termico estivo di tutto il policlinico. Questo comporta un mancato utilizzo dei reflui termici del cogeneratore.

Le torri evaporative sono state sostituite qualche TOT di anni fa a causa di costi di manutenzione eccessivi.
\clearpage
\section{L'edificio 2}
Tutto il corpo di fabbrica è destinato alla \emph{Cardiochirurgia}.

Esso è costituito da 5 edifici:
\begin{itemize}
	\item Corpo A: è l'edificio principale. Di sviluppo longitudinale lungo un asse orientato lungo la direttrice NE--SO, è costituito da 6 piani fuori terra. Contiene le degenze, gli ambulatori, l'Emodinamica al piano terra, l'UTIC (Unità di Terapia Intensiva Coronarica) al primo piano e il blocco operatorio con terapia intensiva al quinto piano. La sua superficie coperta è di \n{4920}{m^2}, per singolo piani è di \n{818}{m^2} e un volume di \n{14760}{m^3} (circa il 3\% dell'intero complesso dell'AOU Federico II).
	\item Corpo B: contiene gli stabulari. Di pianta pressoché quadrata (\n{102}{m^2}) è alto solo 1 piano. 
	\item Corpo C: contiene laboratori e ambulatori. E' di pianta rettangolare e alto solo 1 piano. Estensione di \n{812}{m^2} \textbf{CHIEDI LA DIAGNOSI!!!}
	\item Corpo D: contiene ambulatori, studi medici e laboratori. Di pianta quadrata (\n{217}{m^2}) è alto solo 1 piano. 
	\item Corpo E: contiene ambulatori cardiologici, geriatrici e di medicina interna, vari laboratori e studi medici. È alto solo 1 piano con un'estensione di \n{305}{m^2}.
\end{itemize}

L'edificio 2 preserva tutte le opere edili e alcune impiantistiche realizzate all'epoca della sua costruzione. Non è difficile dedurre, quindi, che allo stato attuale sia i comportamenti estivi e invernali dell'involucro come le efficienze termo-meccaniche dell'impianto idro-aeraulico siano quantomeno inferiori a quelli consigliati dalla norma attuale vigente. 

%\begin{itemize}
%	\item componenti opachi
%	\item componenti trasparenti
%	\item ponti termici???
%	\item definizione dei vari locali
%\end{itemize}
%Metti i risultati con tabelle. 
%Inserisci foto in bianco e nero di alcuni componenti finestrati o criticità.
%Prendi in considerazione l'idea di inserire planimetrie da stampare su A3 da piegare all'interno della tesi. 
\section{L'involucro}
L'involucro dell'Edificio 2, sia quello opaco che quello trasparente, non è cambiato in questi anni quindi non ci sono differenze con le stratigrafie indicate dall'\tit{Ing.}{Corrado Beguinot}.

Segue, quindi, la descrizione numerica dei componenti (opachi e trasparenti) utilizzati come dati di input per il calcolo del fabbisogno energetico e del carico termico (estivo e invernale) dell'edificio stesso.
\subsection{Componenti opachi}
\subsubsection{MURO EXT}
È il componente esterno delle facciate maggiori del Corpo A. \\È caratterizzato esternamente da blocchi di silicalcite alternati dagli infissi. \\Questa tipologia di muro è fittizia poiché si è modellato un componente che nella realtà è caratterizzata da una diversa stratigrafia in senso verticale. Dal punto di vista numerico, quindi, si è effettuata una media ponderale delle varie caratteristiche termo-fisiche in modo tale che il risultato finale sia quanto più possibile veritiero. La parte inferiore è costituita semplicemente da un mattone forato da \n{10}{cm} intonacato internamente ed esternamente; la parte superiore, invece, è caratterizzata dai blocchi di silicalcite. \\ La stratigrafia della parte superiore è (dall'interno verso l'esterno):
\begin{center}
	\begin{tabular}{lcc}
		\toprule
		Componente & Spessore [m] & Conduttività [\si{W/mK}] \\
		\midrule
		Acciao & \num{0.01} & \num{50.0} \\
		Intercapedine d'aria & \num{0.05} & -\\
		CLS & \num{0.35} & \num{1.06} \\
		\bottomrule
	\end{tabular}
\end{center}
Questi i risultati del componente modellato (a valle della media ponderale):
\begin{center}
	\begin{tabular}{lcc}
		\toprule
		Spessore & \num{0.43} & \si{m}\\
		Trasmittanza & \num{1.423} & \trasm\\
		Trasmittanza termica periodica & \num{0.190} & \trasm\\
		\bottomrule
	\end{tabular}
\end{center}
\subsubsection{MURO EXT 200}
È il componente esterno delle scale e del torrino.
\begin{center}
	\begin{tabular}{lcc}
		\toprule
		Componente & Spessore [m] & Conduttività [\si{W/mK}] \\
		\midrule
		Malta di calce-cemento & \num{0.01} & \num{0.90} \\
		CLS & \num{0.18} & \num{1.48}\\
		Malta di calce-cemento & \num{0.01} & \num{0.90} \\
		\bottomrule
	\end{tabular}
\end{center}
Questi i risultati del componente modellato:
\begin{center}
	\begin{tabular}{lcc}
		\toprule
		Spessore & \num{0.20} & \si{m}\\
		Trasmittanza & \num{3.29} & \trasm\\
		Trasmittanza termica periodica & \num{1.71} & \trasm\\
		\bottomrule
	\end{tabular}
\end{center}

\subsubsection{MURO EXT Corpo Basso}
È il componente esterno dei corpi bassi ovvero del Corpo B, C, D ed E.
\begin{center}
	\begin{tabular}{lcc}
		\toprule
		Componente & Spessore [m] & Conduttività [\si{W/mK}] \\
		\midrule
		Malta di calce-cemento & \num{0.01} & \num{0.90} \\
		Mattone forato & \num{0.08} & -\\
		Intercapedine d'aria & \num{0.05} & - \\
		CLS & \num{0.1} & \num{1.91}\\
		\bottomrule
	\end{tabular}
\end{center}
Questi i risultati del componente modellato:
\begin{center}
	\begin{tabular}{lcc}
		\toprule
		Spessore & \num{0.24} & \si{m}\\
		Trasmittanza & \num{1.63} & \trasm\\
		Trasmittanza termica periodica & \num{1.06} & \trasm\\
		\bottomrule
	\end{tabular}
\end{center}
\subsubsection{COPERTURA 1}
È la copertura del Corpo A.\\Già oggetto di interventi passati, le sue caratteristiche termo-fisiche sono così riassunte:
\begin{center}
	\begin{tabular}{lcc}
		\toprule
		Spessore & \num{0.38} & \si{m}\\
		Trasmittanza & \num{0.36} & \trasm\\
		Trasmittanza termica periodica & \num{0.10} & \trasm\\
		\bottomrule
	\end{tabular}
\end{center}
\subsubsection{COPERTURA 2}
È la copertura dei Corpi B, C, D ed E.
\begin{center}
	\begin{tabular}{lcc}
		\toprule
		Componente & Spessore [m] & Conduttività [\si{W/mK}] \\
		\midrule
		Intonaco di Calce e Gesso & \num{0.02} & \num{1.61} \\
		CLS SC  & \num{0.09} & \num{1.48}\\
		CLS SA & \num{0.10} & \num{0.58} \\
		Bitume su carta e cartone & \num{0.0050} & \num{0.23} \\
		\bottomrule
	\end{tabular}
\end{center}
Questi i risultati del componente modellato:
\begin{center}
	\begin{tabular}{lcc}
		\toprule
		Spessore & \num{0.22} & \si{m}\\
		Trasmittanza & \num{2.39} & \trasm\\
		Trasmittanza termica periodica & \num{1.07} & \trasm\\
		\bottomrule
	\end{tabular}
\end{center}
\subsubsection{PAVIMENTO}
È il componente opaco utilizzato per modellare il pavimento dell'Edificio 2 (quindi in comune a tutti i corpi). È bene precisare che questo componente non è a contatto con il terreno in quanto vi sono i locali della sottocentrale nel piano -1.\\
Questi i risultati:
\begin{center}
	\begin{tabular}{lcc}
		\toprule
		Componente & Spessore [m] & Conduttività [\si{W/mK}] \\
		\midrule
		Piastrelle di Ceramica & \num{0.01} & \num{1.30} \\
		CLS SC & \num{0.08} & \num{1.61}\\
		Blocco da solaio & \num{0.22} & - \\
		\bottomrule
	\end{tabular}
\end{center}
\begin{center}
	\begin{tabular}{lcc}
		\toprule
		Spessore & \num{0.31} & \si{m}\\
		Trasmittanza & \num{1.38} & \trasm\\
		Trasmittanza termica periodica & \num{0.35} & \trasm\\
		\bottomrule
	\end{tabular}
\end{center}
\clearpage
\subsection{Componenti trasparenti}
Come è già stato ampiamente detto, tutti gli infissi risultano essere ancora quelli originali.

Per la loro modellazione sono stati usati gli stessi dati termo-fisici ($U_g$, $U_f$ e $U_w$) mentre sono stati differenziati solo geometricamente.
Il telaio è metallico senza taglio termico con un unico vetro (spessore di \n{4}{mm} senza alcun trattamento superficiale).

Questi i risultati:
\begin{center}
	\begin{tabular}{lcc}
		\toprule
		$U_g$ & \num{5.747} & \multirow{3}*{\trasm}\\
		$U_f$ & \num{5.800} &\\
		$U_g$ & \num{5.760} &\\
		\bottomrule
	\end{tabular}
\end{center}

Si elencano ora i vari infissi utilizzati all'interno del modello dell'edificio:
\begin{itemize}
	\item \textbf{PICCOLA} \n{1.60x0.33}{m}.\\ È il componente trasparente facente parte delle facciate maggiori del Corpo~A. 
	\item \textbf{GRANDE} \n{1.60x0.67}{m}.\\ È la variante alta della finestra \textbf{PICCOLA}. 
	\item \textbf{LUCERNARIO} \n{1.45x0.39}{m}.\\ Questo è il lucernario presente nella parte superiore di ogni modulo delle due facciate maggiori del Corpo A. 
	\item \textbf{QUADRA} \n{0.75x0.75}{m}.
	\item \textbf{LUNGA} \n{0.75x1.70}{m}.\\ Presente al di sotto della finestra \textbf{QUADRA}, insieme a quest'ultima crea un unico infisso che percorre tutta l'altezza del Corpo A nelle scanalature del muro \textbf{MURO EXT 200}. 
	\item \textbf{FIN-160} \n{1.60x3.00}{m}. \\ Presente nel corridoio antistante la \emph{medicheria} in ogni piano. Anche questo infisso, come \textbf{FINESTRA LUNGA} e \textbf{FINESTRA QUADRA}, genera un'unica finestra che percorre tutta l'altezza del corpo A.
	\item \textbf{PT-160} \n{1.60x2.00}{m}. \\ È la finestra dei Corpi B, C, D ed E. 
	\item \textbf{PT-Alta Corpo Basso} \n{1.60x0.35}{m}.\\ È il lucernario di ogni modulo caratterizzante i Corpi B, C, D ed E. 
\end{itemize}

A conclusione dell'analisi si evidenzia come tutti i valori di trasmittanza calcolati sia per le strutture opache verticali che quelle orizzontali come anche i componenti trasparenti risultano sensibilmente superiori ai valori limite minimi contenuti in \ref{vallim}.

\subsection{Definizione locali}
Il carico termico di un edificio (come anche il suo fabbisogno energetico) non tiene conto solamente dell'ambiente esterno (temperatura e umidità tutto l'anno mentre la radiazione solare solo in estate). È molto importante considerare la destinazione d'uso del locale che si intende climatizzare. Pertanto sono stati individuati le tipologie di locali e per ognuno di essi si sono definiti i seguenti parametri:
\begin{itemize}
	\item \emph{Temperatura} e \emph{umidità relativa} di progetto nella stagione estiva e invernale. Una loro adeguata scelta in fase progettuale e poi un loro mantenimento ad impianto ultimato e perfettamente funzionante sono alla base del benessere di una persona che vive in un locale;
	\item La \emph{portata di rinnovo} dalla norma \norvent. Rappresenta la quantità di aria esterna che si suppone essere priva di agenti nocivi per l'uomo e che è necessario introdurre all'interno dell'ambiente da climatizzare per mantenere entro certi limiti la qualità, appunto, dell'aria;
	\item Gli \emph{apporti interni di calore}:
	\begin{itemize}
		\item \emph{L'occupazione}, ovvero il numero di persone che affollano il locale con i conseguenti apporti di calore sensibile e latente. In assenza di dati certi (ovvero nell'impossibilità di conoscere le persone che effettivamente affollano un locale - contando il numero di posti a sedere in una sala di un cinema, per esempio) si procede utilizzando i valori di occupazione fissati dalla \norvent;
		\item \emph{Apparati interni}, ovvero i carichi dovuti a macchinari/fonti di calore sensibile e latente presenti all'interno del locale;
		\item \emph{L'illuminazione}, ovvero la quantità di calore sensibile (trasmesso per convezione e irraggiamento) dovuto all'illuminazione;
	\end{itemize}
\end{itemize}
È molto importante notare che per alcuni di questi apporti interni di calore sono stati definiti dei profili d'uso temporali su base oraria. \emph{L'occupazione} dello Studio Medico, per esempio, ha un profilo d'uso del tipo:
\begin{center}
	\begin{tabular}{lr}
		\toprule
		Dalle 00:00 alle 08:00 & 0\% \\
		Dalle 08:00 alle 17:00 & 100\% \\
		Dalle 17:00 alle 00:00 & 0\% \\
		\bottomrule
	\end{tabular}
\end{center}
Ovvero all'interno degli studi medici vi sarà il massimo dell'occupazione (calcolata considerando l'indice di affollamento $n_S$ che in questo caso è pari a \n{0.05}{pers/m^2} tratto sempre dalla \norvent) dalle 8:00 alle 17:00 ogni giorno dell'anno.

Si vogliono descrivere in maniera dettagliata alcune tipologie di locali che sono stati maggiormente utilizzati per la modellazione dell'edificio.
\subsubsection{Degenza}
\texttt{Inserisci una foto esterna per far vedere dove stanno le degenze}.
\begin{center}
	\begin{tabular}{lcc}
										& Raffrescamento 			& Riscaldamento \\
		Temperatura interna di progetto & \n{24.0}{\degreeCelsius} 	& \n{21.0}{\degreeCelsius}\\
		Umidità interna di progetto 	& \n{50.0}{\%}				& \n{30.0}{\%}\\
		\midrule
		Ventilazione					& \multicolumn{2}{c}{\n{11}{l/s} per persona} 		\\
		\midrule
		\multirow{3}*{Occupazione}		& \multicolumn{2}{c}{\n{5}{m^2/pers}}  		\\
									 	& \n{75}{W/pers} 		& sensibile	\\
										& \n{70}{W/pers}		& latente 	\\
		\midrule
		Apparati interni 				&\n{15}{W/m^2}			& sensibile\\
		\midrule
		Illuminazione					& \multicolumn{2}{c}{\n{11.3}{W/m^2}}\\
		\midrule
		Altri carichi					& \multicolumn{2}{c}{--}\\				
	\end{tabular}
\end{center}
\subsubsection{Laboratorio}
\texttt{Inserisci una foto esterna per far vedere dove stanno i laboratori}.
\begin{center}
	\begin{tabular}{lcc}
										& Raffrescamento 			& Riscaldamento \\
		Temperatura interna di progetto & \n{25.0}{\degreeCelsius} 	& \n{21.0}{\degreeCelsius}\\
		Umidità interna di progetto 	& \n{50.0}{\%}				& \n{50.0}{\%}\\
		\midrule
		Ventilazione					& \multicolumn{2}{c}{\n{6}{vol/h}} 		\\
		\midrule
		\multirow{3}*{Occupazione}		& \multicolumn{2}{c}{\n{20}{m^2/pers}}  		\\
										& \n{75}{W/pers} 		& sensibile	\\
										& \n{55}{W/pers}		& latente 	\\
		\midrule
		Apparati interni 				&\n{40}{W/m^2}			& sensibile\\
		\midrule
		Illuminazione					& \multicolumn{2}{c}{\n{11.3}{W/m^2}}\\
		\midrule
		Altri carichi					& \n{1000}{W}			& sensibile \\				
	\end{tabular}
\end{center}
\subsubsection{Studio Medico}
\texttt{Inserisci una foto esterna per far vedere dove stanno gli studi medici}.
\begin{center}
	\begin{tabular}{lcc}
										& Raffrescamento 			& Riscaldamento \\
		Temperatura interna di progetto & \n{25.0}{\degreeCelsius} 	& \n{22.0}{\degreeCelsius}\\
		Umidità interna di progetto 	& \n{50.0}{\%}				& \n{40.0}{\%}\\
		\midrule
		Ventilazione					& \multicolumn{2}{c}{\n{11}{l/s} per persona} 		\\
		\midrule
		\multirow{3}*{Occupazione}		& \multicolumn{2}{c}{\n{4}{pers}}  		\\
										& \n{75}{W/pers} 		& sensibile	\\
										& \n{55}{W/pers}		& latente 	\\
		\midrule
		Apparati interni 				&\multicolumn{2}{c}{--}\\
		\midrule
		Illuminazione					& \multicolumn{2}{c}{\n{11.3}{W/m^2}}\\
		\midrule
		Altri carichi					& \multicolumn{2}{c}{--}\\				
	\end{tabular}
\end{center}
\subsubsection{Cucina}
\texttt{Inserisci una foto esterna per far vedere dove stanno le cucine}.
\begin{center}
	\begin{tabular}{lcc}
										& Raffrescamento 			& Riscaldamento \\
		Temperatura interna di progetto & \n{28.0}{\degreeCelsius} 	& \n{20.0}{\degreeCelsius}\\
		Umidità interna di progetto 	& \n{50.0}{\%}				& \n{30.0}{\%}\\
		\midrule
		Ventilazione					& \multicolumn{2}{c}{\n{16.5}{l/s} per \si{m^2}} 		\\
		\midrule
		\multirow{3}*{Occupazione}		& \multicolumn{2}{c}{\n{5}{m^2/pers}}  		\\
										& \n{75}{W/pers} 		& sensibile	\\
										& \n{70}{W/pers}		& latente 	\\
		\midrule
		Apparati interni 				& \n{5.40}{W/m^2} 		& sensibile \\
		\midrule
		Illuminazione					& \multicolumn{2}{c}{\n{11.3}{W/m^2}}\\
		\midrule
		\multirow{2}{*}{Altri carichi}	& \n{1500}{W} 		& sensibile \\
										& \n{500}{W} 		& latente   \\
	\end{tabular}
\end{center}
\subsubsection{Ufficio}
\texttt{Inserisci una foto esterna per far vedere dove stanno gli uffici}.
\begin{center}
	\begin{tabular}{lcc}
										& Raffrescamento 			& Riscaldamento \\
		Temperatura interna di progetto & \n{25.0}{\degreeCelsius} 	& \n{21.0}{\degreeCelsius}\\
		Umidità interna di progetto 	& \n{50.0}{\%}				& \n{40.0}{\%}\\
		\midrule
		Ventilazione					& \multicolumn{2}{c}{\n{11}{l/s} per persona} 		\\
		\midrule
		\multirow{3}*{Occupazione}		& \multicolumn{2}{c}{\n{10}{m^2/pers}}  		\\
										& \n{75}{W/pers} 		& sensibile	\\
										& \n{70}{W/pers}		& latente 	\\
		\midrule
		Apparati interni 				& \n{15}{W/m^2} 		& sensibile \\
		\midrule
		Illuminazione					& \multicolumn{2}{c}{\n{11.3}{W/m^2}}\\
		\midrule
		Altri carichi					& \multicolumn{2}{c}{--}\\
	\end{tabular}
\end{center}
\section{L'impianto}
L'impianto dell'Edificio 2 è attualmente caratterizzato da una sottocentrale, presente nel piano \num{-1}, la quale alimenta varie unità locali (radiatori e fancoil) e UTA.

Nel primo capitolo è già stato ampiamente detto che i vari edifici del Policlinico sfruttano la rete di presidio di acqua surriscaldata e refrigerata.

Partendo dal lato utenza, il carico sensibile invernale viene coperto da radiatori (si veda la \subref{radiatori} in \vref{unitloca}) presenti all'interno di ogni piano e fancoil solo nel III e IV piano. Il carico estivo (sia sensibile che latente) invece viene coperto da monosplit e fancoil ad acqua montati negli ultimi anni durante varie ristrutturazioni (come nel caso del terzo e quarto piano). In questi due piani vi è anche una rete aeraulica per il rinnovo dell'aria. Nonostante la presenza di questi impianti, non vengono garantiti l'adeguato recupero energetico dall'impianto di ventilazione e il controllo termo-igrometrico dai fancoil e radiatori. Queste carenze sono evidenti nell'eccessivo ricorso a split per il controllo della temperatura (e in modo indiretto dell'umidità) durante la stagione estiva. È evidente la necessità di una riqualificazione. Negli altri 3 piani la ventilazione è garantita tramite infiltrazione naturale (ovvero apertura delle finestre di piano) mentre è assente del tutto il controllo termo-igrometrico nella stagione estiva. Anche in questi piani si è fatto largo uso di monosplit a parete.

\begin{figure}
	\centering
	\subfloat[][\emph{Un tipico radiatore presente nel corridoio del Piano Terra -- Corpo A.}]{\includegraphics[width=0.45\textwidth]{6_2_cap/img/radiatore}\label{radiatori}}	\quad
	\subfloat[][\emph{Corridoio del IV Piano. Si noti il mix di unità locali.}]{\includegraphics[width=0.45\textwidth]{6_2_cap/img/P4RadFan}}	\\
	\subfloat[][\emph{Area Relax del IV Piano. In alto la griglia di mandata del fancoil. Questi hanno anche una presa di aria di rinnovo esterna.}]{\includegraphics[width=0.45\textwidth]{6_2_cap/img/P4FanCoil}} \quad
	\subfloat[][\emph{Diffusore circolare in una degenza del III Piano per l'aria di rinnovo.}]{\includegraphics[width=0.45\textwidth]{6_2_cap/img/P3Aer}}
	\caption{Unità Locali nel Corpo A}\label{unitloca}
\end{figure}
	
	

Il quinto piano (in cui è presente il blocco operatorio di cardiochirurgia) è gestito da un adeguato impianto a tutt'aria presente in copertura.

In una porzione del primo piano vi è l'UTIC (\emph{Unità di Terapia Intensiva Coronarica}) che è trattata da un altro impianto a tutt'aria montato in un locale dello stesso piano. L'UTA dell'Emodinamica (Piano Terra) con la relativa centrale termo-frigorifera è presente all'esterno.

Nonostante siano state elencati e descritti gli usi dei diversi edifici, l'intervento di riqualificazione energetica riguarda solo i corpi C e A esclusi il Reparto di Emodinamica al Piano Terra, l'UTIC al Primo Piano e il Quinto Piano.

Si riportano ora le varie planimetrie dell'Edificio 2 con evidenziate le zone che non saranno interessate dall'intervento.

%\includepdf[pagecommand={\null\vfill\captionof{table}{caioooo}}]{6_2_cap/img/PT.pdf}\label{PT}
\includepdf{6_2_cap/img/PT.pdf}\label{PT}
\includepdf{6_2_cap/img/P1.pdf}\label{P1}
\includepdf{6_2_cap/img/P2.pdf}\label{P2}
\includepdf{6_2_cap/img/P3.pdf}\label{P3}
\includepdf{6_2_cap/img/P4.pdf}\label{P4}

\section{I risultati energetici}
L'analisi di carichi e fabbisogni energetici è stata effettuata con l'ausilio di software basati sulla tecnologia \bim.

Il \bim\ (\emph{Building Information Modelling}) è un metodo che, citando da \textit{Wikipedia.it}, permette:
\begin{quote}
	 \dots l'ottimizzazione della pianificazione, realizzazione e gestione di costruzioni tramite aiuto di un software. Tramite esso tutti i dati rilevanti di una costruzione possono essere raccolti, combinati e collegati digitalmente. La costruzione virtuale è visualizzabile inoltre come un modello geometrico tridimensionale.%\footnote{https://it.wikipedia.org/wiki/Building\_Information\_Modeling}
\end{quote}
Si sta lavorando a livello Europeo affinché questa metodologia di progettazione possa avere una sua definizione. Al giorno d'oggi, infatti, il BIM viene frainteso con una qualche sorta di tecnologia o addirittura software: niente di più sbagliato. Gli applicativi basati sulla \emph{filosofia del BIM} hanno delle peculiarità che li differenziano in modo marcato dagli altri. Infatti in un programma BIM un oggetto (per esempio un edificio) viene rappresentato tridimensionalmente perché disegnato da un architetto ma ogni parte di questo edificio contiene delle informazioni utili ad uno strutturista (ovvero tipologia del materiale usato per un muro o anche le sue caratteristiche fisiche). Continuando in questa direzione, altre informazioni che è possibile inserire in questo edificio (o file) sono le caratteristiche termiche dell'involucro (opaco e trasparente), il disegno/progettazione dei vari impianti (idraulici, aeraulici e elettrici). Una volta inserite tutte queste informazioni è possibile ricavare dei dati molto preziosi. Per esempio il carico termico nella stagione estiva/invernale, il computo metrico del materiale utilizzato, fare stime e/o studi sulla vita utile dell'edificio, etc\dots

Tutte queste informazioni sono presenti all'interno di un unico file. E siccome il file è unico, viene incentivata la cooperazione tra i diversi professionisti. Addirittura è possibile lavorare contemporaneamente su quest'unico file in modo tale che una modifica di una parte del progetto si ripercuote automaticamente sugli altri aspetti progettuali in cosicché le altre figure professionali vengono automaticamente avvertite di suddetta modifica. Questo si traduce in una maggiore velocità di esecuzione, una sostanziale diminuzione di errori e costi di progettazione.

Citando dal sito della ACCA:
\begin{quote}
	Grazie alla metodologia del BIM l'edificio viene "costruito" prima della sua realizzazione fisica, mediante un modello virtuale, attraverso la collaborazione ed i contributi di tutti gli attori coinvolti nel progetto (architetti, ingegneri, progettisti, consulenti, analisti energetici, etc\dots).
\end{quote}
Nello studio di riqualificazione energetica in questione, il programma utilizzato è costituito da una suite di applicativi diversi che si interfacciano in modo tale da realizzare ciò che la metodologia BIM impone.

Di \textbf{CYPE}, ovvero la suite, sono stati usati i seguenti programmi:
\begin{description}
	\item[IFC Builder]permette di disegnare geometricamente l'edificio definendo di volta in volta i suoi elementi (involucro opaco e trasparente, locali e zone termiche). Il file \textsc{.ifc} realizzato permette di essere esportato e utilizzato da qualsiasi altro programma che supporta questo file. Il formato \textsc{.ifc} è aperto, libero e ben documentato. In parole povere: questo file è alla base del BIM in quanto permette la cooperazione tra diversi programmi (usati rispettivamente da diverse figure professionali);
	\item[CYPETHERM Loads]permette di definire le caratteristiche termo-fisiche dell'involucro e la destinazione d'uso dei locali importati dal file \textsc{.ifc}. Restituisce i carichi termici (annuali, mensili e orari) per ogni locale delle zone termiche. Una volta ottenuti i risultati è possibile esportarli in modo tale che qualsiasi altro programma della suite CYPE possa utilizzarli;
	\item[CYPETHERM HVAC]permette di importare i file realizzati con IFC Builder e Loads e progettare/dimensionare un adeguato impianto HVAC per l'abbattimento del carico sensibile e latente oltre che per il rinnovo dell'aria;
\end{description}
In \ref{bim:cype} è riassunto il funzionamento della suite di CYPE. Si nota la centralità del server \emph{BIMserver.center} a cui tutti i progetti prodotti dai vari programmi fanno capo.
\begin{figure}[h]
	\centering
	\caption[Schema BIM CYPE]{Schema riassuntivo del funzionamento della \emph{suite} CYPE con filosofia BIM.}
	\label{bim:cype}
	\includegraphics[width=0.5\textheight]{6_2_cap/img/cypeflow}
\end{figure}

Per quanto riguarda la parte idronica si è utilizzato un applicativo BIM della software-house \emph{C.A.T.S.} che si appoggia ad \emph{Autodesk Autocad}.

È stato possibile definire inizialmente la tipologia di tubazioni da utilizzare (dimensioni, materiale e coibente), la metodologia di dimensionamento con le velocità minime ammissibili e poi disegnare direttamente in Autocad le tubazioni stesse posizionando le unità locali (radiatori e fancoil). Infine, l'applicativo ha dimensionato le tubazioni, rilasciato il computo metrico e la relazione di calcolo.

L'edificio è stato suddiviso in 5 zone termiche:
\begin{itemize}
	\item \textbf{V Piano};
	\item \textbf{UTIC};
	\item \textbf{Emodinamica};
	\item \textbf{Corpo A} (escluse le zone del V Piano, UTIC e Emodinamica);
	\item \textbf{Corpo C};
	\item \textbf{Radiatori}: ovvero tutti i servizi igienici del Corpo A. 
\end{itemize}
È necessario spiegare il motivo di questa suddivisione.

Innanzitutto le prime tre zone termiche sono state separate dal resto del Corpo A in quanto il loro impianto non è oggetto di intervento ma è comunque interessante conoscere il risparmio in potenza che si riesce ad ottenere.

Per quanto riguarda la zona \emph{Radiatori}, è effettivamente sbagliato dal punto di vista termico considerarla esclusa dal resto del Corpo A in quanto non sono due zone termiche distinte. Questa forzatura è stata necessaria in quanto dai risultati ottenuti non era possibile scindere i carichi sensibili dei servizi igienici dal resto dell'edificio. Siccome i radiatori verranno posizionati solo nei servizi igienici, era interessante conoscere, quindi, il carico termico dei soli servizi igienici per poi dimensionare adeguatamente l'impianto a suo servizio. 

\subsection{Stagione Estiva}
Si presentano i vari risultati ottenuti con la dovuta teoria che vi è alla base.

Nel secondo capitolo si è già abbondantemente spiegato il motivo per cui è necessario considerare anche i carichi interni e quelli latenti.

Le valutazioni energetiche sono state effettuate con una temperatura e umidità esterna rispettivamente di \n{35}{\degreeCelsius} con il \n{60}{\%} maggiore di quella consigliata dalla \norvent. All'interno, invece, si sono rispettate le condizioni di benessere termo-igrometrico: \n{24}{\degreeCelsius} con il \n{50}{\%}. Per la ventilazione si è proceduto considerando per ogni locale i ricambi orari minimi.

Relativamente ai soli reparti di Emodinamica, UTIC e Blocco Operatorio che possiedono un loro impianto dedicato a \emph{tutt'aria} si è proceduto in questo modo.
Sono note le superfici e le altezze dei vari locali come anche il carico esterno solare e di trasmissione; si sono supposti gli \num{8} ricambi orari di aria esterna; un affollamento di $n=\SI{0.08}{pers/m^2}$; un carico sensibile e latente per le persone pari rispettivamente a $Q_{pers,S}=\SI{70}{W/pers}$ e $Q_{pers,L}=\SI{50}{W/pers}$; un carico interno per le varie apparecchiature di $Q_{int,S}=\SI{10}{W/m^2}$. 

Nella seguente tabella sono mostrati i parametri di input e i risultati per le 3 zone termiche considerate.
\begin{center}
	\begin{tabular}{lccccc}
		\toprule
					&	Superficie 				&	Portata Aria Est. 			&	$\dot{Q}_{sol}+\dot{Q}_{trasm}$		& 	\multirow{2}{*}{RST}		&	$\dot{Q}_{frigo}$ 	\\
					&	{\small \si{m^2}}		&		{\small \si{kg/s}}		&		{\small \si{kW}}				&								&{\small \si{kW}}		\\					
		\midrule	
		UTIC		&		\num{146}			&		\num{1.46}				&	\num{7.62}		&	0.94					&	\num{10.5}		\\
		Emod.		&		\num{169}			&		\num{1.69}				&	\num{13.2}		&	0.96					&	\num{16.5}		\\
		B.O.		&	\num{697}				&		\num{7.33}				&	\num{34.6}		&	0.94					&	\num{48.3}		\\
		\bottomrule
	\end{tabular}
\end{center}
RST sta per \emph{Rapporto Sensibile su Totale} ed è adimensionale. È molto importante nei diagrammi psicrometrici in quanto permette di individuare i luoghi dei punti di immissione dell'aria (nel caso di impianti a tutt'aria o di miscelazione in quelli misti) che bilanciano adeguatamente sia il carico sensibile che latente. Siccome nel nostro caso il suo valore è prossimo all'unità avremo un carico da bilanciare che è pressoché tutto sensibile.

Conoscendo il valore di RST, del carico da bilanciare $Q_{frigo}$ e della portata di ventilazione, dal diagramma psicrometrico si individuano, per ogni zona termica (e le loro rispettive UTA), le proprietà termoigrometriche dell'aria da immettere. Ed è, quindi, possibile calcolare le potenze delle batterie di raffreddamento e post-riscaldamento delle UTA nel caso estivo. Nella seguente tabella sono riassunti i risultati ottenuti in \si{kW}. L'ipotesi alla base di tutte queste valutazioni è che le batterie siano ideali cioè che non vi sia una frazione di portata d'aria che non entri a contatto con le lamelle delle batterie stesse. In questa sede, però, volendo solo effettuare una valutazione pre e post intervento non è importante considerare anche il cosiddetto \emph{by-pass factor} delle batterie (anche se oggigiorno suddetti fattori risultano essere dell'ordine del \n{5}{\%} e quindi tranquillamente trascurabili).
\begin{center}
	\begin{tabular}{lcc}
		&	$Q_{UTA,f}$		&	$Q_{UTA,c}$\\
		\midrule
		UTIC	&	\num{79.3}			&	\num{6.86}\\
		Emod.	&	\num{91.7}			&	\num{3.52}\\
		B.O.	&	\num{664}			&	\num{39.0}\\
	\end{tabular}
\end{center}
La potenza di raffreddamento e riscaldamento viene fornita da una opportuna portata d'acqua prodotta in sottocentrale. Una rapida valutazione delle portate in gioco, restituisce i valori, in \si{l/s}, riportati nella tabella successiva. Per questioni tecnologiche le temperature di mandata e ritorno dell'acqua sono rispettivamente di \num{7} e \n{12}{\degreeCelsius} per la batteria di raffreddamento mentre di \num{60} e \n{55}{\degreeCelsius} per quelle di riscaldamento e che il calore specifico dell'acqua è ritenuto costante e pari a $c_p=\SI{4.2}{kJ/kgK}$.
\begin{center}
	\begin{tabular}{lcc}
			&	$\dot{m}_f$	&	$\dot{m}_c$\\
			\midrule
			UTIC	&	\num{3.8}	&	\num{0.33}\\
			Emod.	&	\num{4.4}	&	\num{0.17}\\
			B.O.	&	\num{32}		&	\num{1.8}\\
	\end{tabular}
\end{center}

Per quanto riguarda \corpa\ e \corpc, l'impianto utilizzato è di tipo \emph{misto} (come nel III e IV piano) o addirittura assente. Pertanto non ne è stata effettuata una valutazione sulla ventilazione ma i risultati considerano solo la trasmissione tramite l'involucro e i carichi interni: apparecchiature e persone.

Per la zona termica \radd\ è ovvio che il carico di raffreddamento sia nullo.

In questa tabella sono riassunti tutti i risultati ottenuti per la stagione estiva nello stato di fatto. 
%{Per i radiatori ovviamente la potenza di raffrescamento è nulla. Per UTIC, BO e EMO il totale tiene conto della trasmissione (ottenuta da CYPE) mentre i carichi interni (sensibili e latenti) sono stati calcolati su Excel. Poi tramite diagramma psicrometrico ho calcolato anche la potenza di ventilazione. Per i due corpi A e C i valori sono stati presi da CYPE senza passare per Excel e per questo motivo i valori tengono conto anche della ventilazione. Tutti i valori stanno su excel.}
\begin{center}
	\small
	\begin{tabular}{lccccc}
		\toprule
		\multirow{2}*{Zona Termica} & Superficie 		& Ventilazione 					& $Q_L$ 			& $Q_S$ 				& Totale \\
									& \si{m^2}		& \si{m^3/h}						& \si{kW}			& \si{kW}					& \si{kW}\\
		\midrule
		Radiatori		& \num{291.1}				& ---								& ---				& ---						& ---\\
		B.O.			& \num{696.7}				& \num{21936}						& \num{2.78}		& \num{45.5}				& \num{673.3}\\
		UTIC			& \num{145.7}				& \num{4371}						& \num{0.583}		& \num{9.89}				& \num{82.9}\\
		Emod.			& \num{164.4}				& \num{5058}						& \num{0.674}		& \num{15.9} 				& \num{104.7}\\
		Corpo C			& \num{529.2}				& --								& \num{37.6}		& \num{106}					& \num{144}\\
		Corpo A			& \num{2352.2}				& --								& \num{72.6}		& \num{246}					& \num{319}\\
		\bottomrule
	\end{tabular}
\end{center}

Si ricordi che i carichi sensibili e latenti per la stagione estiva sono definiti come segue:
\begin{align}
	\dot{Q}_S	&=\dot{Q}_{sol}+\dot{Q}_{trasm}+\dot{Q}_{int,S}+\dot{Q}_{pers,S}\\
	\dot{Q}_L	&=\dot{Q}_{int,L}+\dot{Q}_{pers,L}
\end{align}
\subsection{Stagione Invernale}
Come è già stato detto nel secondo capitolo, in questa stagione si trascura qualsiasi apporto di carico latente. Le valutazioni sono state fatte con una temperatura esterna di \n{0}{\degreeCelsius} mentre all'interno vengono garantiti i \n{22}{\degreeCelsius} col \n{45}{\%} di umidità relativa.

Per determinare le condizioni termoigrometriche di immissione dell'aria nell'UTIC, Emodinamica e nel Blocco Operatorio si è proceduto facendo un bilancio di energia:
\begin{align}
\dot{m}h_s	&	=\dot{m}h_i+\dot{Q}_{trasm}		\label{bilancio}\\
h_s			&	=h_i+\frac{\dot{Q}_{trasm}}{\dot{m}}	\notag\\
t_s			&	=\frac{t_ic_p\dot{m}+\dot{Q}_{trasm}}{c_p\dot{m}}\label{bilancio2}
\end{align}
Nell'equazione \vref{bilancio2} si è supposta l'aria secca trascurando, quindi, il contributo del vapore d'acqua.

Con l'ausilio del diagramma psicrometrico, si è proceduto a valutare la potenza necessaria per abbattere il carico invernale. Sono note queste quantità: la portata di ventilazione (che si suppone uguale a quella estiva), il carico invernale da abbattere, le condizioni interne e esterne, il valore di RST. Intersecando le varie rette corrispondenti alle altrettante trasformazioni si sono ottenuti i risultati ricercati riassunti nella seguente tabella. Ovviamente la potenza termica risultante è la somma di quella necessaria alla batteria di pre e post riscaldamento.
\begin{center}
	\label{UTA:potenzaFATTO}
	\begin{tabular}{lccc}
		
						& RTS					&	$t_s$ [\si{\degreeCelsius}]		&	kW	\\
		\midrule
		UTIC			&\multirow{3}{*}{1.1}	&	\num{23.3}						&	\num{52.5}\\
		Emod.			&						&	\num{27.4}						&	\num{67.4}\\
		B.O.			&						&	\num{25.2}						&	\num{279}\\
	\end{tabular}
\end{center}
In definitiva, i risultati sono riportati di seguito per ogni zona termica. Anche in questo caso, per i corpi A e C non si è tenuto conto della ventilazione.
%{Per i Radiatori il totale tiene conto solo della trasmissione. Per BO, UTIC e EMO ho fatto la somma tra la trasmissione e la potenza di ventilazione calcolata tramite diagramma psicrometrico e ma i risultati stanno anche sul file excel. Per i due Corpi A e C il totale è solo la trasmissione in quanto non è presente la ventilazione.}
\begin{center}
	\small
	\begin{tabular}{lcccc}
		\toprule
		\multirow{2}*{Zona Termica} & Superficie 		& Trasmissione 					& Ventilazione 	 			& Totale		\\
									& [\si{m^2}]		& [\si{kW}]						& [\si{m^3/h}]				& [\si{kW}]		\\
		\midrule
		Radiatori					& \num{291.1}		& \num{30.7}					& ---						& \num{30.7}	\\
		B.O.						& \num{696.7}		& \num{38.4}					& \num{21936}				& \num{317}	\\
		UTIC						& \num{145.7}		& \num{4.75}					& \num{4371}				& \num{57.3}	\\
		Emod.						& \num{164.4}		& \num{12.5}					& \num{5058}				& \num{79.9} 	\\
		Corpo C						& \num{529.2}		& \num{59.9}					& --						& \num{59.9}	\\
		Corpo A						& \num{2352.2}		& \num{142}						& --						& \num{142}	\\
		\bottomrule
	\end{tabular}
\end{center}
La potenza totale tiene conto sia della trasmissione che della ventilazione calcolata nella tabella precedente.