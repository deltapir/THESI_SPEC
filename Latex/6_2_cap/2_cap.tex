\chapter{Lo stato attuale}
\thispagestyle{empty}

Aggiungi un mini indice in questo capitolo se le dimensioni dello stesso iniziano ad essere proibitive.\\
Parla anche dell'assenza di organi per la misurazione dei consumi (elettrici e termici) e quindi problemi per la valutazione energetica \emph{ante-operam} e per il rispetto ai requisiti \emph{CAM}.\vspace{1em}

Il presente elaborato di laurea si riferisce ad alcuni corpi dell'edificio 2. Quest'ultimo ospita UTIC, Emodinamica e i relativi laboratori. Complessivamente, compresi i servizi generali, gli ambulatori, gli uffici e gli spazi di circolazione, la superficie totale è pari a \num{100000}\si{m^2} \SI{100000}{m^2} \n{10101010}{m_2}
\begin{itemize}
	\item componenti opachi
	\item componenti trasparenti
	\item ponti termici???
	\item definizione dei vari locali
\end{itemize}
Metti i risultati con tabelle. 
Inserisci foto in bianco e nero di alcuni componenti finestrati o criticità.
Prendi in considerazione l'idea di inserire planimetrie da stampare su A3 da piegare all'interno della tesi. 
\section{L'involucro}
\subsection{Componenti opachi}
\subsection{Componenti trasparenti}
\subsection{Definizione locali (norma 10339)}
\section{L'impianto}
\section{I risultati energetici}
\subsection{Stagione Estiva}
Inserisci sia le variazioni dei carichi termici e dell'indice di prestazione energetica.
\subsection{Stagione Invernale}
